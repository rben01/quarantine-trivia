\documentclass[11pt]{beamer}
\usepackage{graphicx}
\usepackage[export]{adjustbox}
\usepackage{ifthen}
\usepackage{xeCJK}
\usepackage[T1]{fontenc}
\usepackage{xfrac}

\usetheme[hideothersubsections]{Goettingen}
\usecolortheme{seahorse}
\setbeamercovered{invisible}
\setbeamertemplate{navigation symbols}{\insertslidenavigationsymbol}
\setbeamertemplate{page number in head/foot}{}
\setbeamertemplate{blocks}[rounded][shadow=false]
% \setbeamerfont{section in sidebar}{size=\fontsize{4}{3}\selectfont}
% \setbeamerfont{subsection in sidebar}{size=\fontsize{4}{3}\selectfont}
% \setbeamerfont{subsubsection in sidebar}{size=\fontsize{4}{2}\selectfont}

\usepackage{microtype}
% \DisableLigatures[f]{encoding = *, family = *}

% \usefonttheme{professionalfonts} % using non standard fonts for beamer
\usefonttheme{serif} % default family is serif
\usepackage{XCharter}
% stix2
% XCharter
% (sans) [defaultsans]{cantarell}

\AtBeginSection[]{
  \begin{frame}
    \vfill
    \centering
    \begin{beamercolorbox}[sep=8pt,center,shadow=true,rounded=true]{title}
    \usebeamerfont{title}\insertsectionhead\par%
    \ifthenelse{\equal{\thisSectionName}{Bonus}}{}{
        \usebeamerfont{subtitle}\thisSectionName\par%
    }
    \end{beamercolorbox}
    \begin{center}
    Please mute yourselves!
    \end{center}

    \ifthenelse{\equal{\thisSectionName}{Bonus}}
    {
        Get ready for some \emph{devilishly} hard questions!
        \vspace*{1em}
        \includegraphics[max width=0.5\textwidth,
            max height=0.4\textheight]{Images/devil.jpg}
    }{}

    \ifthenelse{\equal{\thisSectionName}{Ancient Civilizations}}
    {
        \includegraphics[max width=0.5\textwidth,
            max height=0.4\textheight]{Images/belushi.jpg}
    }{}

    \vfill
  \end{frame}
}

\AtBeginSubsection[]{
  \begin{frame}
    \vfill
    \centering
    \begin{beamercolorbox}[sep=8pt,center,shadow=true,rounded=true]{title}
    \usebeamerfont{title}\insertsectionhead\par%
    \usebeamerfont{subtitle}\insertsubsectionhead\par%
    \end{beamercolorbox}
    \ifthenelse{\equal{\subsecname}{Answers}} {
        \begin{center}
        Unmute yourselves!
        \end{center}
    }
    \vfill
  \end{frame}
}
\begin{document}

\title{Welcome to Quarantine Trivia V!\vspace{-0.5in}}
\date{}

\begin{frame}
\titlepage{}
\begin{center}
\includegraphics[max width=0.9\textwidth,
    max height=0.4\textheight]{Images/triviatitleframelogo.png}
\end{center}
\end{frame}

\begingroup{}
\begin{frame}
\vfill{}
\begin{beamercolorbox}[sep=8pt,center,shadow=true,rounded=true]{title}
\usebeamerfont{title}Good luck everyone! And have fun!
\end{beamercolorbox}
\vfill{}
\end{frame}
\endgroup{}
\def\thisSectionName{Logos}
\section{Round 1}
\subsection*{Q1}
\begin{frame}[t]{Logos, Question 1}
% \vspace{0.5em}
\begin{columns}[T,totalwidth=\linewidth]
\begin{column}{0.32\linewidth}
\begin{block}{Question}
This is the former logo of a baking company. What is the company, which used to be known as NBC, known as today?
\end{block}
\end{column}
\begin{column}{0.65\linewidth}
\begin{center}
\includegraphics[max width=0.95\textwidth,max height=0.7\textheight]{{Images/nabisco}.jpg}
\end{center}
\end{column}
\end{columns}
\end{frame}
\subsection*{Q2}
\begin{frame}[t]{Logos, Question 2}
% \vspace{0.5em}
\begin{columns}[T,totalwidth=\linewidth]
\begin{column}{0.32\linewidth}
\begin{block}{Question}
Which college's logo is this?
\end{block}
\end{column}
\begin{column}{0.65\linewidth}
\begin{center}
\includegraphics[max width=0.95\textwidth,max height=0.7\textheight]{{Images/cornell}.png}
\end{center}
\end{column}
\end{columns}
\end{frame}
\subsection*{Q3}
\begin{frame}[t]{Logos, Question 3}
% \vspace{0.5em}
\begin{columns}[T,totalwidth=\linewidth]
\begin{column}{0.32\linewidth}
\begin{block}{Question}
The image here was cropped from which company's logo?
\end{block}
\end{column}
\begin{column}{0.65\linewidth}
\begin{center}
\includegraphics[max width=0.95\textwidth,max height=0.7\textheight]{{Images/hboicon}.png}
\end{center}
\end{column}
\end{columns}
\end{frame}
\subsection*{Q4}
\begin{frame}[t]{Logos, Question 4}
% \vspace{0.5em}
\begin{columns}[T,totalwidth=\linewidth]
\begin{column}{0.32\linewidth}
\begin{block}{Question}
The logo here, which has had its text removed, is which car company's logo?
\end{block}
\end{column}
\begin{column}{0.65\linewidth}
\begin{center}
\includegraphics[max width=0.95\textwidth,max height=0.7\textheight]{{Images/fiatnotext}.png}
\end{center}
\end{column}
\end{columns}
\end{frame}
\subsection*{Q5}
\begin{frame}[t]{Logos, Question 5}
% \vspace{0.5em}
\begin{columns}[T,totalwidth=\linewidth]
\begin{column}{0.32\linewidth}
\begin{block}{Question}
In 2016, which company announced that this would be their new logo?
\end{block}
\end{column}
\begin{column}{0.65\linewidth}
\begin{center}
\includegraphics[max width=0.95\textwidth,max height=0.7\textheight]{{Images/hplogo}.jpg}
\end{center}
\end{column}
\end{columns}
\end{frame}
\subsection*{Q6}
\begin{frame}[t]{Logos, Question 6}
% \vspace{0.5em}
\begin{columns}[T,totalwidth=\linewidth]
\begin{column}{0.32\linewidth}
\begin{block}{Question}
Which company's logo is this?
\end{block}
\end{column}
\begin{column}{0.65\linewidth}
\begin{center}
\includegraphics[max width=0.95\textwidth,max height=0.7\textheight]{{Images/morton}.png}
\end{center}
\end{column}
\end{columns}
\end{frame}
\subsection*{Q7}
\begin{frame}[t]{Logos, Question 7}
% \vspace{0.5em}
\begin{columns}[T,totalwidth=\linewidth]
\begin{column}{0.32\linewidth}
\begin{block}{Question}
Which company's logo is this?
\end{block}
\end{column}
\begin{column}{0.65\linewidth}
\begin{center}
\includegraphics[max width=0.95\textwidth,max height=0.7\textheight]{{Images/atari}.jpg}
\end{center}
\end{column}
\end{columns}
\end{frame}
\subsection*{Q8}
\begin{frame}[t]{Logos, Question 8}
% \vspace{0.5em}
\begin{columns}[T,totalwidth=\linewidth]
\begin{column}{0.32\linewidth}
\begin{block}{Question}
The text has been removed from this logo. Which company's logo is it?
\end{block}
\end{column}
\begin{column}{0.65\linewidth}
\begin{center}
\includegraphics[max width=0.95\textwidth,max height=0.7\textheight]{{Images/levisicon}.png}
\end{center}
\end{column}
\end{columns}
\end{frame}
\subsection*{Q9}
\begin{frame}[t]{Logos, Question 9}
% \vspace{0.5em}
\begin{columns}[T,totalwidth=\linewidth]
\begin{column}{0.32\linewidth}
\begin{block}{Question}
Which company's logo is this?
\end{block}
\end{column}
\begin{column}{0.65\linewidth}
\begin{center}
\includegraphics[max width=0.95\textwidth,max height=0.7\textheight]{{Images/reebokicon}.jpg}
\end{center}
\end{column}
\end{columns}
\end{frame}
\subsection*{Q10}
\begin{frame}[t]{Logos, Question 10}
% \vspace{0.5em}
\begin{columns}[T,totalwidth=\linewidth]
\begin{column}{0.32\linewidth}
\begin{block}{Question}
This apostrophe is from the logo of which well-known food and drink chain?
\end{block}
\end{column}
\begin{column}{0.65\linewidth}
\begin{center}
\includegraphics[max width=0.95\textwidth,max height=0.7\textheight]{{Images/ddicon}.png}
\end{center}
\end{column}
\end{columns}
\end{frame}
\subsection{Answers}
\begin{frame}[t]{Logos, Answer 1}
% \vspace{0.5em}
\begin{columns}[T,totalwidth=\linewidth]
\begin{column}{0.32\linewidth}
\begin{block}{Question}
This is the former logo of a baking company. What is the company, which used to be known as NBC, known as today?
\end{block}
\visible<2->{
    \begin{block}{Answer}
    Nabisco
    \end{block}
}
\end{column}
\begin{column}{0.65\linewidth}
\begin{center}
\includegraphics[max width=0.95\textwidth,max height=0.7\textheight]{{Images/nabisco}.jpg}
\end{center}
\end{column}
\end{columns}
\end{frame}
\begin{frame}[t]{Logos, Answer 2}
% \vspace{0.5em}
\begin{columns}[T,totalwidth=\linewidth]
\begin{column}{0.32\linewidth}
\begin{block}{Question}
Which college's logo is this?
\end{block}
\visible<2->{
    \begin{block}{Answer}
    Cornell
    \end{block}
}
\end{column}
\begin{column}{0.65\linewidth}
\begin{center}
\includegraphics[max width=0.95\textwidth,max height=0.7\textheight]{{Images/cornell}.png}
\end{center}
\end{column}
\end{columns}
\end{frame}
\begin{frame}[t]{Logos, Answer 3}
% \vspace{0.5em}
\begin{columns}[T,totalwidth=\linewidth]
\begin{column}{0.32\linewidth}
\begin{block}{Question}
The image here was cropped from which company's logo?
\end{block}
\end{column}
\begin{column}{0.65\linewidth}
\begin{center}
\includegraphics[max width=0.95\textwidth,max height=0.35\textheight]{{Images/hboicon}.png}
\end{center}
\end{column}
\end{columns}

\visible<2->{
    \begin{columns}[T,totalwidth=\linewidth]
    \begin{column}{0.32\linewidth}
    \begin{block}{Answer}
    HBO
    \end{block}
    \end{column}
    \begin{column}{0.65\linewidth}
    \begin{center}
    \includegraphics[max width=0.95\textwidth,
        max height=0.38\textheight]{{Images/hbologo}.png}
    \end{center}
    \end{column}
    \end{columns}
}
\end{frame}
\begin{frame}[t]{Logos, Answer 4}
% \vspace{0.5em}
\begin{columns}[T,totalwidth=\linewidth]
\begin{column}{0.32\linewidth}
\begin{block}{Question}
The logo here, which has had its text removed, is which car company's logo?
\end{block}
\end{column}
\begin{column}{0.65\linewidth}
\begin{center}
\includegraphics[max width=0.95\textwidth,max height=0.35\textheight]{{Images/fiatnotext}.png}
\end{center}
\end{column}
\end{columns}

\visible<2->{
    \begin{columns}[T,totalwidth=\linewidth]
    \begin{column}{0.32\linewidth}
    \begin{block}{Answer}
    Fiat
    \end{block}
    \end{column}
    \begin{column}{0.65\linewidth}
    \begin{center}
    \includegraphics[max width=0.95\textwidth,
        max height=0.38\textheight]{{Images/fiatlogo}.png}
    \end{center}
    \end{column}
    \end{columns}
}
\end{frame}
\begin{frame}[t]{Logos, Answer 5}
% \vspace{0.5em}
\begin{columns}[T,totalwidth=\linewidth]
\begin{column}{0.32\linewidth}
\begin{block}{Question}
In 2016, which company announced that this would be their new logo?
\end{block}
\visible<2->{
    \begin{block}{Answer}
    Hewlett-Packard
    \end{block}
}
\end{column}
\begin{column}{0.65\linewidth}
\begin{center}
\includegraphics[max width=0.95\textwidth,max height=0.7\textheight]{{Images/hplogo}.jpg}
\end{center}
\end{column}
\end{columns}
\end{frame}
\begin{frame}[t]{Logos, Answer 6}
% \vspace{0.5em}
\begin{columns}[T,totalwidth=\linewidth]
\begin{column}{0.32\linewidth}
\begin{block}{Question}
Which company's logo is this?
\end{block}
\visible<2->{
    \begin{block}{Answer}
    Morton Salt
    \end{block}
}
\end{column}
\begin{column}{0.65\linewidth}
\begin{center}
\includegraphics[max width=0.95\textwidth,max height=0.7\textheight]{{Images/morton}.png}
\end{center}
\end{column}
\end{columns}
\end{frame}
\begin{frame}[t]{Logos, Answer 7}
% \vspace{0.5em}
\begin{columns}[T,totalwidth=\linewidth]
\begin{column}{0.32\linewidth}
\begin{block}{Question}
Which company's logo is this?
\end{block}
\visible<2->{
    \begin{block}{Answer}
    Atari
    \end{block}
}
\end{column}
\begin{column}{0.65\linewidth}
\begin{center}
\includegraphics[max width=0.95\textwidth,max height=0.7\textheight]{{Images/atari}.jpg}
\end{center}
\end{column}
\end{columns}
\end{frame}
\begin{frame}[t]{Logos, Answer 8}
% \vspace{0.5em}
\begin{columns}[T,totalwidth=\linewidth]
\begin{column}{0.32\linewidth}
\begin{block}{Question}
The text has been removed from this logo. Which company's logo is it?
\end{block}
\end{column}
\begin{column}{0.65\linewidth}
\begin{center}
\includegraphics[max width=0.95\textwidth,max height=0.35\textheight]{{Images/levisicon}.png}
\end{center}
\end{column}
\end{columns}

\visible<2->{
    \begin{columns}[T,totalwidth=\linewidth]
    \begin{column}{0.32\linewidth}
    \begin{block}{Answer}
    Levi's
    \end{block}
    \end{column}
    \begin{column}{0.65\linewidth}
    \begin{center}
    \includegraphics[max width=0.95\textwidth,
        max height=0.38\textheight]{{Images/levislogo}.png}
    \end{center}
    \end{column}
    \end{columns}
}
\end{frame}
\begin{frame}[t]{Logos, Answer 9}
% \vspace{0.5em}
\begin{columns}[T,totalwidth=\linewidth]
\begin{column}{0.32\linewidth}
\begin{block}{Question}
Which company's logo is this?
\end{block}
\end{column}
\begin{column}{0.65\linewidth}
\begin{center}
\includegraphics[max width=0.95\textwidth,max height=0.35\textheight]{{Images/reebokicon}.jpg}
\end{center}
\end{column}
\end{columns}

\visible<2->{
    \begin{columns}[T,totalwidth=\linewidth]
    \begin{column}{0.32\linewidth}
    \begin{block}{Answer}
    Reebok
    \end{block}
    \end{column}
    \begin{column}{0.65\linewidth}
    \begin{center}
    \includegraphics[max width=0.95\textwidth,
        max height=0.38\textheight]{{Images/reeboklogo}.jpg}
    \end{center}
    \end{column}
    \end{columns}
}
\end{frame}
\begin{frame}[t]{Logos, Answer 10}
% \vspace{0.5em}
\begin{columns}[T,totalwidth=\linewidth]
\begin{column}{0.32\linewidth}
\begin{block}{Question}
This apostrophe is from the logo of which well-known food and drink chain?
\end{block}
\end{column}
\begin{column}{0.65\linewidth}
\begin{center}
\includegraphics[max width=0.95\textwidth,max height=0.35\textheight]{{Images/ddicon}.png}
\end{center}
\end{column}
\end{columns}

\visible<2->{
    \begin{columns}[T,totalwidth=\linewidth]
    \begin{column}{0.32\linewidth}
    \begin{block}{Answer}
    Dunkin' / Dunkin' Donuts
    \end{block}
    \end{column}
    \begin{column}{0.65\linewidth}
    \begin{center}
    \includegraphics[max width=0.95\textwidth,
        max height=0.38\textheight]{{Images/ddlogo}.png}
    \end{center}
    \end{column}
    \end{columns}
}
\end{frame}
\def\thisSectionName{Cocktails}
\section{Round 2}
\subsection*{Q1}
\begin{frame}[t]{Cocktails, Question 1}
% \vspace{0.5em}
\begin{block}{Question}
\begin{itemize}
\item 2\({}^1{\mskip -5mu⁄\mskip -3mu}_2\) oz.\ rye whiskey
\item 1 oz.\ sweet vermouth
\item 1 dash Angostura bitters
\item Garnish with a Maraschino cherry
\item Serve straight up
\end{itemize}
\end{block}
\end{frame}
\subsection*{Q2}
\begin{frame}[t]{Cocktails, Question 2}
% \vspace{0.5em}
\begin{block}{Question}
\begin{itemize}
\item 1\({}^1{\mskip -5mu⁄\mskip -3mu}_2\) oz.\ vodka
\item 1 dash lime juice
\item 4 oz.\ ginger beer
\item Garnish with a slice of lime
\item Serve over ice in a copper mug
\end{itemize}
\end{block}
\end{frame}
\subsection*{Q3}
\begin{frame}[t]{Cocktails, Question 3}
% \vspace{0.5em}
\begin{block}{Question}
\begin{itemize}
\item 1 oz.\ gin
\item 2 dashes simple syrup
\item \({}^1{\mskip -5mu⁄\mskip -3mu}_2\) oz.\ lemon juice
\item 2 oz.\ champagne (chilled)
\item Garnish with lemon peel
\item Serve in a champagne glass
\end{itemize}
\end{block}
\end{frame}
\subsection*{Q4}
\begin{frame}[t]{Cocktails, Question 4}
% \vspace{0.5em}
\begin{block}{Question}
\begin{itemize}
\item 2 oz.\ cognac
\item 2 dashes Peychaud's bitters
\item 1 dash absinthe
\item 1 sugar cube
\item Garnish with lemon peel
\item Serve straight up
\end{itemize}
\end{block}
\end{frame}
\subsection*{Q5}
\begin{frame}[t]{Cocktails, Question 5}
% \vspace{0.5em}
\begin{block}{Question}
\begin{itemize}
\item 1\({}^1{\mskip -5mu⁄\mskip -3mu}_2\) oz.\ gin
\item 1\({}^1{\mskip -5mu⁄\mskip -3mu}_2\) oz.\ sweet red vermouth
\item 1\({}^1{\mskip -5mu⁄\mskip -3mu}_2\) oz.\ Campari
\item Garnish with orange slice or peel
\item Serve over ice
\end{itemize}
\end{block}
\end{frame}
\subsection*{Q6}
\begin{frame}[t]{Cocktails, Question 6}
% \vspace{0.5em}
\begin{block}{Question}
\begin{itemize}
\item 1\({}^1{\mskip -5mu⁄\mskip -3mu}_2\) oz.\ bourbon
\item 1 oz.\ lemon juice
\item \({}^1{\mskip -5mu⁄\mskip -3mu}_2\) oz.\ simple syrup
\item (Optional) 1 dash egg white
\item Garnish with Maraschino cherry and orange slice
\item Serve straight up or on the rocks
\end{itemize}
\end{block}
\end{frame}
\subsection*{Q7}
\begin{frame}[t]{Cocktails, Question 7}
% \vspace{0.5em}
\begin{block}{Question}
\begin{itemize}
\item 1\({}^1{\mskip -5mu⁄\mskip -3mu}_2\) oz.\ gin
\item 1 oz.\ fresh lemon juice
\item \({}^1{\mskip -5mu⁄\mskip -3mu}_2\) oz.\ Gomme syrup
\item 2\({}^1{\mskip -5mu⁄\mskip -3mu}_2\) oz.\ soda water
\item Serve on the rocks
\end{itemize}
\end{block}
\end{frame}
\subsection*{Q8}
\begin{frame}[t]{Cocktails, Question 8}
% \vspace{0.5em}
\begin{block}{Question}
\begin{itemize}
\item 1\({}^1{\mskip -5mu⁄\mskip -3mu}_2\) oz.\ vodka
\item 4 oz.\ cranberry juice
\item 1 oz.\ grapefruit juice
\item Garnish with a slice of lime
\item Serve on the rocks
\end{itemize}
\end{block}
\end{frame}
\subsection*{Q9}
\begin{frame}[t]{Cocktails, Question 9}
% \vspace{0.5em}
\begin{block}{Question}
\begin{itemize}
\item 1\({}^1{\mskip -5mu⁄\mskip -3mu}_2\) oz.\ white rum
\item 1 oz.\ dark rum
\item \({}^1{\mskip -5mu⁄\mskip -3mu}_2\) oz.\ orange curaçao
\item \({}^1{\mskip -5mu⁄\mskip -3mu}_2\) oz.\ orgeat syrup
\item 1 dash fresh lime juice
\item Garnish with lime wedge and mint leaves
\item Serve on the rocks
\end{itemize}
\end{block}
\end{frame}
\subsection*{Q10}
\begin{frame}[t]{Cocktails, Question 10}
% \vspace{0.5em}
\begin{block}{Question}
\begin{itemize}
\item \({}^1{\mskip -5mu⁄\mskip -3mu}_2\) oz.\ crème de menthe
\item \({}^1{\mskip -5mu⁄\mskip -3mu}_2\) oz.\ crème de cacao
\item \({}^1{\mskip -5mu⁄\mskip -3mu}_2\) oz.\ cream
\item Serve in a chilled cocktail glass
\end{itemize}
\end{block}
\end{frame}
\subsection{Answers}
\begin{frame}[t]{Cocktails, Answer 1}
% \vspace{0.5em}
\begin{block}{Question}
\begin{itemize}
\item 2\({}^1{\mskip -5mu⁄\mskip -3mu}_2\) oz.\ rye whiskey
\item 1 oz.\ sweet vermouth
\item 1 dash Angostura bitters
\item Garnish with a Maraschino cherry
\item Serve straight up
\end{itemize}
\end{block}

\visible<2->{
    \begin{columns}[T,totalwidth=\linewidth]
    \begin{column}{0.32\linewidth}
    \begin{block}{Answer}
    A Manhattan
    \end{block}
    \end{column}
    \begin{column}{0.65\linewidth}
    \begin{center}
    \includegraphics[max width=0.95\textwidth,
        max height=0.30000\textheight]{{Images/manhattan}.jpg}
    \end{center}
    \end{column}
    \end{columns}
}
\end{frame}
\begin{frame}[t]{Cocktails, Answer 2}
% \vspace{0.5em}
\begin{block}{Question}
\begin{itemize}
\item 1\({}^1{\mskip -5mu⁄\mskip -3mu}_2\) oz.\ vodka
\item 1 dash lime juice
\item 4 oz.\ ginger beer
\item Garnish with a slice of lime
\item Serve over ice in a copper mug
\end{itemize}
\end{block}

\visible<2->{
    \begin{columns}[T,totalwidth=\linewidth]
    \begin{column}{0.32\linewidth}
    \begin{block}{Answer}
    Moscow Mule
    \end{block}
    \end{column}
    \begin{column}{0.65\linewidth}
    \begin{center}
    \includegraphics[max width=0.95\textwidth,
        max height=0.30000\textheight]{{Images/moscowmule}.jpg}
    \end{center}
    \end{column}
    \end{columns}
}
\end{frame}
\begin{frame}[t]{Cocktails, Answer 3}
% \vspace{0.5em}
\begin{block}{Question}
\begin{itemize}
\item 1 oz.\ gin
\item 2 dashes simple syrup
\item \({}^1{\mskip -5mu⁄\mskip -3mu}_2\) oz.\ lemon juice
\item 2 oz.\ champagne (chilled)
\item Garnish with lemon peel
\item Serve in a champagne glass
\end{itemize}
\end{block}

\visible<2->{
    \begin{columns}[T,totalwidth=\linewidth]
    \begin{column}{0.32\linewidth}
    \begin{block}{Answer}
    French 75
    \end{block}
    \end{column}
    \begin{column}{0.65\linewidth}
    \begin{center}
    \includegraphics[max width=0.95\textwidth,
        max height=0.26000\textheight]{{Images/french75}.jpg}
    \end{center}
    \end{column}
    \end{columns}
}
\end{frame}
\begin{frame}[t]{Cocktails, Answer 4}
% \vspace{0.5em}
\begin{block}{Question}
\begin{itemize}
\item 2 oz.\ cognac
\item 2 dashes Peychaud's bitters
\item 1 dash absinthe
\item 1 sugar cube
\item Garnish with lemon peel
\item Serve straight up
\end{itemize}
\end{block}

\visible<2->{
    \begin{columns}[T,totalwidth=\linewidth]
    \begin{column}{0.32\linewidth}
    \begin{block}{Answer}
    Sazerac
    \end{block}
    \end{column}
    \begin{column}{0.65\linewidth}
    \begin{center}
    \includegraphics[max width=0.95\textwidth,
        max height=0.30000\textheight]{{Images/sazerac}.jpg}
    \end{center}
    \end{column}
    \end{columns}
}
\end{frame}
\begin{frame}[t]{Cocktails, Answer 5}
% \vspace{0.5em}
\begin{block}{Question}
\begin{itemize}
\item 1\({}^1{\mskip -5mu⁄\mskip -3mu}_2\) oz.\ gin
\item 1\({}^1{\mskip -5mu⁄\mskip -3mu}_2\) oz.\ sweet red vermouth
\item 1\({}^1{\mskip -5mu⁄\mskip -3mu}_2\) oz.\ Campari
\item Garnish with orange slice or peel
\item Serve over ice
\end{itemize}
\end{block}

\visible<2->{
    \begin{columns}[T,totalwidth=\linewidth]
    \begin{column}{0.32\linewidth}
    \begin{block}{Answer}
    Negroni
    \end{block}
    \end{column}
    \begin{column}{0.65\linewidth}
    \begin{center}
    \includegraphics[max width=0.95\textwidth,
        max height=0.22000\textheight]{{Images/negroni}.jpg}
    \end{center}
    \end{column}
    \end{columns}
}
\end{frame}
\begin{frame}[t]{Cocktails, Answer 6}
% \vspace{0.5em}
\begin{block}{Question}
\begin{itemize}
\item 1\({}^1{\mskip -5mu⁄\mskip -3mu}_2\) oz.\ bourbon
\item 1 oz.\ lemon juice
\item \({}^1{\mskip -5mu⁄\mskip -3mu}_2\) oz.\ simple syrup
\item (Optional) 1 dash egg white
\item Garnish with Maraschino cherry and orange slice
\item Serve straight up or on the rocks
\end{itemize}
\end{block}

\visible<2->{
    \begin{columns}[T,totalwidth=\linewidth]
    \begin{column}{0.32\linewidth}
    \begin{block}{Answer}
    Whiskey Sour or Boston Sour
    \end{block}
    \end{column}
    \begin{column}{0.65\linewidth}
    \begin{center}
    \includegraphics[max width=0.95\textwidth,
        max height=0.18000\textheight]{{Images/whiskeysour}.jpg}
    \end{center}
    \end{column}
    \end{columns}
}
\end{frame}
\begin{frame}[t]{Cocktails, Answer 7}
% \vspace{0.5em}
\begin{block}{Question}
\begin{itemize}
\item 1\({}^1{\mskip -5mu⁄\mskip -3mu}_2\) oz.\ gin
\item 1 oz.\ fresh lemon juice
\item \({}^1{\mskip -5mu⁄\mskip -3mu}_2\) oz.\ Gomme syrup
\item 2\({}^1{\mskip -5mu⁄\mskip -3mu}_2\) oz.\ soda water
\item Serve on the rocks
\end{itemize}
\end{block}

\visible<2->{
    \begin{columns}[T,totalwidth=\linewidth]
    \begin{column}{0.32\linewidth}
    \begin{block}{Answer}
    Gin fizz
    \end{block}
    \end{column}
    \begin{column}{0.65\linewidth}
    \begin{center}
    \includegraphics[max width=0.95\textwidth,
        max height=0.22000\textheight]{{Images/ginfizz}.jpg}
    \end{center}
    \end{column}
    \end{columns}
}
\end{frame}
\begin{frame}[t]{Cocktails, Answer 8}
% \vspace{0.5em}
\begin{block}{Question}
\begin{itemize}
\item 1\({}^1{\mskip -5mu⁄\mskip -3mu}_2\) oz.\ vodka
\item 4 oz.\ cranberry juice
\item 1 oz.\ grapefruit juice
\item Garnish with a slice of lime
\item Serve on the rocks
\end{itemize}
\end{block}

\visible<2->{
    \begin{columns}[T,totalwidth=\linewidth]
    \begin{column}{0.32\linewidth}
    \begin{block}{Answer}
    Sea Breeze
    \end{block}
    \end{column}
    \begin{column}{0.65\linewidth}
    \begin{center}
    \includegraphics[max width=0.95\textwidth,
        max height=0.30000\textheight]{{Images/seabreeze}.jpg}
    \end{center}
    \end{column}
    \end{columns}
}
\end{frame}
\begin{frame}[t]{Cocktails, Answer 9}
% \vspace{0.5em}
\begin{block}{Question}
\begin{itemize}
\item 1\({}^1{\mskip -5mu⁄\mskip -3mu}_2\) oz.\ white rum
\item 1 oz.\ dark rum
\item \({}^1{\mskip -5mu⁄\mskip -3mu}_2\) oz.\ orange curaçao
\item \({}^1{\mskip -5mu⁄\mskip -3mu}_2\) oz.\ orgeat syrup
\item 1 dash fresh lime juice
\item Garnish with lime wedge and mint leaves
\item Serve on the rocks
\end{itemize}
\end{block}

\visible<2->{
    \begin{columns}[T,totalwidth=\linewidth]
    \begin{column}{0.32\linewidth}
    \begin{block}{Answer}
    Mai Tai
    \end{block}
    \end{column}
    \begin{column}{0.65\linewidth}
    \begin{center}
    \includegraphics[max width=0.95\textwidth,
        max height=0.14000\textheight]{{Images/maitai}.jpg}
    \end{center}
    \end{column}
    \end{columns}
}
\end{frame}
\begin{frame}[t]{Cocktails, Answer 10}
% \vspace{0.5em}
\begin{block}{Question}
\begin{itemize}
\item \({}^1{\mskip -5mu⁄\mskip -3mu}_2\) oz.\ crème de menthe
\item \({}^1{\mskip -5mu⁄\mskip -3mu}_2\) oz.\ crème de cacao
\item \({}^1{\mskip -5mu⁄\mskip -3mu}_2\) oz.\ cream
\item Serve in a chilled cocktail glass
\end{itemize}
\end{block}

\visible<2->{
    \begin{columns}[T,totalwidth=\linewidth]
    \begin{column}{0.32\linewidth}
    \begin{block}{Answer}
    Grasshopper
    \end{block}
    \end{column}
    \begin{column}{0.65\linewidth}
    \begin{center}
    \includegraphics[max width=0.95\textwidth,
        max height=0.26000\textheight]{{Images/grasshopper}.jpg}
    \end{center}
    \end{column}
    \end{columns}
}
\end{frame}
\def\thisSectionName{Superheroes}
\section{Round 3}
\subsection*{Q1}
\begin{frame}[t]{Superheroes, Question 1}
% \vspace{0.5em}
\begin{block}{Question}
What is the name of the mugger who murdered young Bruce Wayne's parents?
\end{block}
\end{frame}
\subsection*{Q2}
\begin{frame}[t]{Superheroes, Question 2}
% \vspace{0.5em}
\begin{block}{Question}
Name all three actors who have played Spider-Man on the big screen in the past twenty years.
\end{block}
\end{frame}
\subsection*{Q3}
\begin{frame}[t]{Superheroes, Question 3}
% \vspace{0.5em}
\begin{block}{Question}
Stan Lee based Tony Stark on which real-life person who was, among other things, a business magnate, a film director and a pilot?
\end{block}
\end{frame}
\subsection*{Q4}
\begin{frame}[t]{Superheroes, Question 4}
% \vspace{0.5em}
\begin{block}{Question}
Psychologist William Moulton Marston, who invented one of the first polygraph machines, also created which well-known superhero?
\end{block}
\end{frame}
\subsection*{Q5}
\begin{frame}[t]{Superheroes, Question 5}
% \vspace{0.5em}
\begin{block}{Question}
Who was the very first DC Comics superhero?
\end{block}
\end{frame}
\subsection*{Q6}
\begin{frame}[t]{Superheroes, Question 6}
% \vspace{0.5em}
\begin{block}{Question}
Which superhero founded The Avengers and occasionally travels by goat-drawn chariot?
\end{block}
\end{frame}
\subsection*{Q7}
\begin{frame}[t]{Superheroes, Question 7}
% \vspace{0.5em}
\begin{block}{Question}
What is Superman's Kryptonian birth name?
\end{block}
\end{frame}
\subsection*{Q8}
\begin{frame}[t]{Superheroes, Question 8}
% \vspace{0.5em}
\begin{block}{Question}
Who was the very first African-American superhero?
\end{block}
\end{frame}
\subsection*{Q9}
\begin{frame}[t]{Superheroes, Question 9}
% \vspace{0.5em}
\begin{block}{Question}
Thankfully, this 2008 superhero movie was successful at the box office; otherwise the titular character might have gotten angry.  What was the three-word title of the movie?
\end{block}
\end{frame}
\subsection*{Q10}
\begin{frame}[t]{Superheroes, Question 10}
% \vspace{0.5em}
\begin{block}{Question}
In \emph{Superman \#199}, a race was held between Superman and which other superhero to determine who was fastest? (The race ended in a tie.)
\end{block}
\end{frame}
\subsection{Answers}
\begin{frame}[t]{Superheroes, Answer 1}
% \vspace{0.5em}
\begin{block}{Question}
What is the name of the mugger who murdered young Bruce Wayne's parents?
\end{block}

\visible<2->{
    \begin{columns}[T,totalwidth=\linewidth]
    \begin{column}{0.32\linewidth}
    \begin{block}{Answer}
    Joe Chill
    \end{block}
    \end{column}
    \begin{column}{0.65\linewidth}
    \begin{center}
    \includegraphics[max width=0.95\textwidth,
        max height=0.54000\textheight]{{Images/joechill}.png}
    \end{center}
    \end{column}
    \end{columns}
}
\end{frame}
\begin{frame}[t]{Superheroes, Answer 2}
% \vspace{0.5em}
\begin{block}{Question}
Name all three actors who have played Spider-Man on the big screen in the past twenty years.
\end{block}

\visible<2->{
    \begin{columns}[T,totalwidth=\linewidth]
    \begin{column}{0.32\linewidth}
    \begin{block}{Answer}
    Toby Maguire, Andrew Garfield, and Tom Holland
    \end{block}
    \end{column}
    \begin{column}{0.65\linewidth}
    \begin{center}
    \includegraphics[max width=0.95\textwidth,
        max height=0.54000\textheight]{{Images/SpiderMan}.jpg}
    \end{center}
    \end{column}
    \end{columns}
}
\end{frame}
\begin{frame}[t]{Superheroes, Answer 3}
% \vspace{0.5em}
\begin{block}{Question}
Stan Lee based Tony Stark on which real-life person who was, among other things, a business magnate, a film director and a pilot?
\end{block}

\visible<2->{
    \begin{columns}[T,totalwidth=\linewidth]
    \begin{column}{0.32\linewidth}
    \begin{block}{Answer}
    Howard Hughes
    \end{block}
    \end{column}
    \begin{column}{0.65\linewidth}
    \begin{center}
    \includegraphics[max width=0.95\textwidth,
        max height=0.50000\textheight]{{Images/hughes4}.jpg}
    \end{center}
    \end{column}
    \end{columns}
}
\end{frame}
\begin{frame}[t]{Superheroes, Answer 4}
% \vspace{0.5em}
\begin{block}{Question}
Psychologist William Moulton Marston, who invented one of the first polygraph machines, also created which well-known superhero?
\end{block}

\visible<2->{
    \begin{columns}[T,totalwidth=\linewidth]
    \begin{column}{0.32\linewidth}
    \begin{block}{Answer}
    Wonder Woman
    \end{block}
    \end{column}
    \begin{column}{0.65\linewidth}
    \begin{center}
    \includegraphics[max width=0.95\textwidth,
        max height=0.50000\textheight]{{Images/wonderwoman2}.jpg}
    \end{center}
    \end{column}
    \end{columns}
}
\end{frame}
\begin{frame}[t]{Superheroes, Answer 5}
% \vspace{0.5em}
\begin{block}{Question}
Who was the very first DC Comics superhero?
\end{block}

\visible<2->{
    \begin{columns}[T,totalwidth=\linewidth]
    \begin{column}{0.32\linewidth}
    \begin{block}{Answer}
    Superman
    \end{block}
    \end{column}
    \begin{column}{0.65\linewidth}
    \begin{center}
    \includegraphics[max width=0.95\textwidth,
        max height=0.58000\textheight]{{Images/supermanactioncomics}.jpeg}
    \end{center}
    \end{column}
    \end{columns}
}
\end{frame}
\begin{frame}[t]{Superheroes, Answer 6}
% \vspace{0.5em}
\begin{block}{Question}
Which superhero founded The Avengers and occasionally travels by goat-drawn chariot?
\end{block}

\visible<2->{
    \begin{columns}[T,totalwidth=\linewidth]
    \begin{column}{0.32\linewidth}
    \begin{block}{Answer}
    Thor
    \end{block}
    \end{column}
    \begin{column}{0.65\linewidth}
    \begin{center}
    \includegraphics[max width=0.95\textwidth,
        max height=0.54000\textheight]{{Images/goats}.jpeg}
    \end{center}
    \end{column}
    \end{columns}
}
\end{frame}
\begin{frame}[t]{Superheroes, Answer 7}
% \vspace{0.5em}
\begin{block}{Question}
What is Superman's Kryptonian birth name?
\end{block}

\visible<2->{
    \begin{columns}[T,totalwidth=\linewidth]
    \begin{column}{0.32\linewidth}
    \begin{block}{Answer}
    Kal-El
    \end{block}
    \end{column}
    \begin{column}{0.65\linewidth}
    \begin{center}
    \includegraphics[max width=0.95\textwidth,
        max height=0.58000\textheight]{{Images/superman}.jpg}
    \end{center}
    \end{column}
    \end{columns}
}
\end{frame}
\begin{frame}[t]{Superheroes, Answer 8}
% \vspace{0.5em}
\begin{block}{Question}
Who was the very first African-American superhero?
\end{block}

\visible<2->{
    \begin{columns}[T,totalwidth=\linewidth]
    \begin{column}{0.32\linewidth}
    \begin{block}{Answer}
    Black Panther (1966)
    \end{block}
    \end{column}
    \begin{column}{0.65\linewidth}
    \begin{center}
    \includegraphics[max width=0.95\textwidth,
        max height=0.58000\textheight]{{Images/blackpanther}.jpg}
    \end{center}
    \end{column}
    \end{columns}
}
\end{frame}
\begin{frame}[t]{Superheroes, Answer 9}
% \vspace{0.5em}
\begin{block}{Question}
Thankfully, this 2008 superhero movie was successful at the box office; otherwise the titular character might have gotten angry.  What was the three-word title of the movie?
\end{block}

\visible<2->{
    \begin{columns}[T,totalwidth=\linewidth]
    \begin{column}{0.32\linewidth}
    \begin{block}{Answer}
    \emph{The Incredible Hulk}
    \end{block}
    \end{column}
    \begin{column}{0.65\linewidth}
    \begin{center}
    \includegraphics[max width=0.95\textwidth,
        max height=0.46000\textheight]{{Images/hulk}.jpg}
    \end{center}
    \end{column}
    \end{columns}
}
\end{frame}
\begin{frame}[t]{Superheroes, Answer 10}
% \vspace{0.5em}
\begin{block}{Question}
In \emph{Superman \#199}, a race was held between Superman and which other superhero to determine who was fastest? (The race ended in a tie.)
\end{block}

\visible<2->{
    \begin{columns}[T,totalwidth=\linewidth]
    \begin{column}{0.32\linewidth}
    \begin{block}{Answer}
    The Flash
    \end{block}
    \end{column}
    \begin{column}{0.65\linewidth}
    \begin{center}
    \includegraphics[max width=0.95\textwidth,
        max height=0.50000\textheight]{{Images/theflash}.jpg}
    \end{center}
    \end{column}
    \end{columns}
}
\end{frame}
\def\thisSectionName{New York City}
\section{Round 4}
\subsection*{Q1}
\begin{frame}[t]{New York City, Question 1}
% \vspace{0.5em}
\begin{block}{Question}
Which modern-day New York periodical did Alexander Hamilton establish in 1801?
\end{block}
\end{frame}
\subsection*{Q2}
\begin{frame}[t]{New York City, Question 2}
% \vspace{0.5em}
\begin{block}{Question}
In which Financial District building was George Washington sworn in as president?
\end{block}
\end{frame}
\subsection*{Q3}
\begin{frame}[t]{New York City, Question 3}
% \vspace{0.5em}
\begin{block}{Question}
What hotel was demolished to build the Empire State Building?
\end{block}
\end{frame}
\subsection*{Q4}
\begin{frame}[t]{New York City, Question 4}
% \vspace{0.5em}
\begin{block}{Question}
Which ballpark did the Brooklyn Dodgers play in before moving to Los Angeles?
\end{block}
\end{frame}
\subsection*{Q5}
\begin{frame}[t]{New York City, Question 5}
% \vspace{0.5em}
\begin{columns}[T,totalwidth=\linewidth]
\begin{column}{0.32\linewidth}
\begin{block}{Question}
Which long-gone building is pictured here?
\end{block}
\end{column}
\begin{column}{0.65\linewidth}
\begin{center}
\includegraphics[max width=0.95\textwidth,max height=0.7\textheight]{{Images/singerbuilding}.jpg}
\end{center}
\end{column}
\end{columns}
\end{frame}
\subsection*{Q6}
\begin{frame}[t]{New York City, Question 6}
% \vspace{0.5em}
\begin{block}{Question}
What is the official name of the Statue of Liberty? (Hint: It is not ``the Statue of Liberty.'')
\end{block}
\end{frame}
\subsection*{Q7}
\begin{frame}[t]{New York City, Question 7}
% \vspace{0.5em}
\begin{block}{Question}
The two most recently-held NYC ticker-tape parades both celebrated victory in the same competition in different years. What was the competition?
\end{block}
\end{frame}
\subsection*{Q8}
\begin{frame}[t]{New York City, Question 8}
% \vspace{0.5em}
\begin{block}{Question}
To within five years, what year did Lincoln Center open?
\end{block}
\end{frame}
\subsection*{Q9}
\begin{frame}[t]{New York City, Question 9}
% \vspace{0.5em}
\begin{columns}[T,totalwidth=\linewidth]
\begin{column}{0.32\linewidth}
\begin{block}{Question}
What are the names of the two lion statues guarding the entrance to the main New York Public Library building at 42\textsuperscript{nd} St.\ and Fifth Avenue?
\end{block}
\end{column}
\begin{column}{0.65\linewidth}
\begin{center}
\includegraphics[max width=0.95\textwidth,max height=0.7\textheight]{{Images/nypl}.jpg}
\end{center}
\end{column}
\end{columns}
\end{frame}
\subsection*{Q10}
\begin{frame}[t]{New York City, Question 10}
% \vspace{0.5em}
\begin{block}{Question}
What non-human mammal is on the seal of the City of New York?
\end{block}
\end{frame}
\subsection{Answers}
\begin{frame}[t]{New York City, Answer 1}
% \vspace{0.5em}
\begin{block}{Question}
Which modern-day New York periodical did Alexander Hamilton establish in 1801?
\end{block}

\visible<2->{
    \begin{columns}[T,totalwidth=\linewidth]
    \begin{column}{0.32\linewidth}
    \begin{block}{Answer}
    \emph{The New York Post} (how far it has fallen!)
    \end{block}
    \end{column}
    \begin{column}{0.65\linewidth}
    \begin{center}
    \includegraphics[max width=0.95\textwidth,
        max height=0.54000\textheight]{{Images/nypost}.jpg}
    \end{center}
    \end{column}
    \end{columns}
}
\end{frame}
\begin{frame}[t]{New York City, Answer 2}
% \vspace{0.5em}
\begin{block}{Question}
In which Financial District building was George Washington sworn in as president?
\end{block}

\visible<2->{
    \begin{columns}[T,totalwidth=\linewidth]
    \begin{column}{0.32\linewidth}
    \begin{block}{Answer}
    Federal Hall
    \end{block}
    \end{column}
    \begin{column}{0.65\linewidth}
    \begin{center}
    \includegraphics[max width=0.95\textwidth,
        max height=0.54000\textheight]{{Images/fedhall}.jpg}
    \end{center}
    \end{column}
    \end{columns}
}
\end{frame}
\begin{frame}[t]{New York City, Answer 3}
% \vspace{0.5em}
\begin{block}{Question}
What hotel was demolished to build the Empire State Building?
\end{block}

\visible<2->{
    \begin{columns}[T,totalwidth=\linewidth]
    \begin{column}{0.32\linewidth}
    \begin{block}{Answer}
    The Waldorf Astoria
    \end{block}
    \end{column}
    \begin{column}{0.65\linewidth}
    \begin{center}
    \includegraphics[max width=0.95\textwidth,
        max height=0.54000\textheight]{{Images/waldorf}.jpg}
    \end{center}
    \end{column}
    \end{columns}
}
\end{frame}
\begin{frame}[t]{New York City, Answer 4}
% \vspace{0.5em}
\begin{block}{Question}
Which ballpark did the Brooklyn Dodgers play in before moving to Los Angeles?
\end{block}

\visible<2->{
    \begin{columns}[T,totalwidth=\linewidth]
    \begin{column}{0.32\linewidth}
    \begin{block}{Answer}
    Ebbets Field
    \end{block}
    \end{column}
    \begin{column}{0.65\linewidth}
    \begin{center}
    \includegraphics[max width=0.95\textwidth,
        max height=0.54000\textheight]{{Images/ebbets}.jpg}
    \end{center}
    \end{column}
    \end{columns}
}
\end{frame}
\begin{frame}[t]{New York City, Answer 5}
% \vspace{0.5em}
\begin{columns}[T,totalwidth=\linewidth]
\begin{column}{0.32\linewidth}
\begin{block}{Question}
Which long-gone building is pictured here?
\end{block}
\visible<2->{
    \begin{block}{Answer}
    The Singer Building (demolition began in 1967, was completed in 1968)
    \end{block}
}
\end{column}
\begin{column}{0.65\linewidth}
\begin{center}
\includegraphics[max width=0.95\textwidth,max height=0.7\textheight]{{Images/singerbuilding}.jpg}
\end{center}
\end{column}
\end{columns}
\end{frame}
\begin{frame}[t]{New York City, Answer 6}
% \vspace{0.5em}
\begin{block}{Question}
What is the official name of the Statue of Liberty? (Hint: It is not ``the Statue of Liberty.'')
\end{block}

\visible<2->{
    \begin{columns}[T,totalwidth=\linewidth]
    \begin{column}{0.32\linewidth}
    \begin{block}{Answer}
    Liberty Enlightening the World
    \end{block}
    \end{column}
    \begin{column}{0.65\linewidth}
    \begin{center}
    \includegraphics[max width=0.95\textwidth,
        max height=0.54000\textheight]{{Images/statueofliberty}.jpg}
    \end{center}
    \end{column}
    \end{columns}
}
\end{frame}
\begin{frame}[t]{New York City, Answer 7}
% \vspace{0.5em}
\begin{block}{Question}
The two most recently-held NYC ticker-tape parades both celebrated victory in the same competition in different years. What was the competition?
\end{block}

\visible<2->{
    \begin{columns}[T,totalwidth=\linewidth]
    \begin{column}{0.32\linewidth}
    \begin{block}{Answer}
    The FIFA Women's World Cup (2015 and 2019)
    \end{block}
    \end{column}
    \begin{column}{0.65\linewidth}
    \begin{center}
    \includegraphics[max width=0.95\textwidth,
        max height=0.50000\textheight]{{Images/tickertape}.jpg}
    \end{center}
    \end{column}
    \end{columns}
}
\end{frame}
\begin{frame}[t]{New York City, Answer 8}
% \vspace{0.5em}
\begin{block}{Question}
To within five years, what year did Lincoln Center open?
\end{block}

\visible<2->{
    \begin{columns}[T,totalwidth=\linewidth]
    \begin{column}{0.32\linewidth}
    \begin{block}{Answer}
    1959 (1954 to 1964 will be accepted)
    \end{block}
    \end{column}
    \begin{column}{0.65\linewidth}
    \begin{center}
    \includegraphics[max width=0.95\textwidth,
        max height=0.54000\textheight]{{Images/lincolncenter}.jpg}
    \end{center}
    \end{column}
    \end{columns}
}
\end{frame}
\begin{frame}[t]{New York City, Answer 9}
% \vspace{0.5em}
\begin{columns}[T,totalwidth=\linewidth]
\begin{column}{0.32\linewidth}
\begin{block}{Question}
What are the names of the two lion statues guarding the entrance to the main New York Public Library building at 42\textsuperscript{nd} St.\ and Fifth Avenue?
\end{block}
\visible<2->{
    \begin{block}{Answer}
    Patience and Fortitude
    \end{block}
}
\end{column}
\begin{column}{0.65\linewidth}
\begin{center}
\includegraphics[max width=0.95\textwidth,max height=0.7\textheight]{{Images/nypl}.jpg}
\end{center}
\end{column}
\end{columns}
\end{frame}
\begin{frame}[t]{New York City, Answer 10}
% \vspace{0.5em}
\begin{block}{Question}
What non-human mammal is on the seal of the City of New York?
\end{block}

\visible<2->{
    \begin{columns}[T,totalwidth=\linewidth]
    \begin{column}{0.32\linewidth}
    \begin{block}{Answer}
    Beaver or beavers
    \end{block}
    \end{column}
    \begin{column}{0.65\linewidth}
    \begin{center}
    \includegraphics[max width=0.95\textwidth,
        max height=0.54000\textheight]{{Images/nycseal}.png}
    \end{center}
    \end{column}
    \end{columns}
}
\end{frame}
\def\thisSectionName{More Plants and Animals}
\section{Round 5}
\subsection*{Q1}
\begin{frame}[t]{More Plants and Animals, Question 1}
% \vspace{0.5em}
\begin{columns}[T,totalwidth=\linewidth]
\begin{column}{0.32\linewidth}
\begin{block}{Question}
What is the name of the fleshy protuberances, pictured here, that grow on the sides of some male orangutans' heads?
\end{block}
\end{column}
\begin{column}{0.65\linewidth}
\begin{center}
\includegraphics[max width=0.95\textwidth,max height=0.7\textheight]{{Images/orangutan}.jpg}
\end{center}
\end{column}
\end{columns}
\end{frame}
\subsection*{Q2}
\begin{frame}[t]{More Plants and Animals, Question 2}
% \vspace{0.5em}
\begin{block}{Question}
Which plant is the primary food source of monarch butterfly caterpillars?
\end{block}
\end{frame}
\subsection*{Q3}
\begin{frame}[t]{More Plants and Animals, Question 3}
% \vspace{0.5em}
\begin{block}{Question}
Biologist J. B. S. Haldane, when asked what could be learned about God from creation, said that God must have an ``inordinate fondness'' for which type of animal, because there are so many different kinds of them?
\end{block}
\end{frame}
\subsection*{Q4}
\begin{frame}[t]{More Plants and Animals, Question 4}
% \vspace{0.5em}
\begin{columns}[T,totalwidth=\linewidth]
\begin{column}{0.32\linewidth}
\begin{block}{Question}
What species of salamander is pictured here?
\end{block}
\end{column}
\begin{column}{0.65\linewidth}
\begin{center}
\includegraphics[max width=0.95\textwidth,max height=0.7\textheight]{{Images/axolotl}.jpg}
\end{center}
\end{column}
\end{columns}
\end{frame}
\subsection*{Q5}
\begin{frame}[t]{More Plants and Animals, Question 5}
% \vspace{0.5em}
\begin{block}{Question}
Which species of bear has the highest proportion of meat in its diet?
\end{block}
\end{frame}
\subsection*{Q6}
\begin{frame}[t]{More Plants and Animals, Question 6}
% \vspace{0.5em}
\begin{block}{Question}
The aardwolf is not a wolf at all, but actually belongs to which family of African mammals?
\end{block}
\end{frame}
\subsection*{Q7}
\begin{frame}[t]{More Plants and Animals, Question 7}
% \vspace{0.5em}
\begin{block}{Question}
What are the two types of vascular tissue responsible for transporting water and nutrients in plants?
\end{block}
\end{frame}
\subsection*{Q8}
\begin{frame}[t]{More Plants and Animals, Question 8}
% \vspace{0.5em}
\begin{block}{Question}
What is the specific term for the chick of a hawk or falcon?
\end{block}
\end{frame}
\subsection*{Q9}
\begin{frame}[t]{More Plants and Animals, Question 9}
% \vspace{0.5em}
\begin{block}{Question}
What polymer is the primary component of the exoskeletons of insects and other arthropods?
\end{block}
\end{frame}
\subsection*{Q10}
\begin{frame}[t]{More Plants and Animals, Question 10}
% \vspace{0.5em}
\begin{block}{Question}
Snails, octopuses, and oysters all belong to which phylum?
\end{block}
\end{frame}
\subsection{Answers}
\begin{frame}[t]{More Plants and Animals, Answer 1}
% \vspace{0.5em}
\begin{columns}[T,totalwidth=\linewidth]
\begin{column}{0.32\linewidth}
\begin{block}{Question}
What is the name of the fleshy protuberances, pictured here, that grow on the sides of some male orangutans' heads?
\end{block}
\visible<2->{
    \begin{block}{Answer}
    Flanges
    \end{block}
}
\end{column}
\begin{column}{0.65\linewidth}
\begin{center}
\includegraphics[max width=0.95\textwidth,max height=0.7\textheight]{{Images/orangutan}.jpg}
\end{center}
\end{column}
\end{columns}
\end{frame}
\begin{frame}[t]{More Plants and Animals, Answer 2}
% \vspace{0.5em}
\begin{block}{Question}
Which plant is the primary food source of monarch butterfly caterpillars?
\end{block}

\visible<2->{
    \begin{columns}[T,totalwidth=\linewidth]
    \begin{column}{0.32\linewidth}
    \begin{block}{Answer}
    Milkweed (\emph{Asclepias})
    \end{block}
    \end{column}
    \begin{column}{0.65\linewidth}
    \begin{center}
    \includegraphics[max width=0.95\textwidth,
        max height=0.54000\textheight]{{Images/milkweed}.jpg}
    \end{center}
    \end{column}
    \end{columns}
}
\end{frame}
\begin{frame}[t]{More Plants and Animals, Answer 3}
% \vspace{0.5em}
\begin{block}{Question}
Biologist J. B. S. Haldane, when asked what could be learned about God from creation, said that God must have an ``inordinate fondness'' for which type of animal, because there are so many different kinds of them?
\end{block}

\visible<2->{
    \begin{columns}[T,totalwidth=\linewidth]
    \begin{column}{0.32\linewidth}
    \begin{block}{Answer}
    Beetles (350,000 species)
    \end{block}
    \end{column}
    \begin{column}{0.65\linewidth}
    \begin{center}
    \includegraphics[max width=0.95\textwidth,
        max height=0.42000\textheight]{{Images/beetles}.jpg}
    \end{center}
    \end{column}
    \end{columns}
}
\end{frame}
\begin{frame}[t]{More Plants and Animals, Answer 4}
% \vspace{0.5em}
\begin{columns}[T,totalwidth=\linewidth]
\begin{column}{0.32\linewidth}
\begin{block}{Question}
What species of salamander is pictured here?
\end{block}
\visible<2->{
    \begin{block}{Answer}
    Axolotl (\emph{Ambystoma mexicanum})
    \end{block}
}
\end{column}
\begin{column}{0.65\linewidth}
\begin{center}
\includegraphics[max width=0.95\textwidth,max height=0.7\textheight]{{Images/axolotl}.jpg}
\end{center}
\end{column}
\end{columns}
\end{frame}
\begin{frame}[t]{More Plants and Animals, Answer 5}
% \vspace{0.5em}
\begin{block}{Question}
Which species of bear has the highest proportion of meat in its diet?
\end{block}

\visible<2->{
    \begin{columns}[T,totalwidth=\linewidth]
    \begin{column}{0.32\linewidth}
    \begin{block}{Answer}
    The polar bear (\emph{Ursus maritimus})
    \end{block}
    \end{column}
    \begin{column}{0.65\linewidth}
    \begin{center}
    \includegraphics[max width=0.95\textwidth,
        max height=0.54000\textheight]{{Images/polarbear}.jpg}
    \end{center}
    \end{column}
    \end{columns}
}
\end{frame}
\begin{frame}[t]{More Plants and Animals, Answer 6}
% \vspace{0.5em}
\begin{block}{Question}
The aardwolf is not a wolf at all, but actually belongs to which family of African mammals?
\end{block}

\visible<2->{
    \begin{columns}[T,totalwidth=\linewidth]
    \begin{column}{0.32\linewidth}
    \begin{block}{Answer}
    Hyenas (\emph{Hyaenidae})
    \end{block}
    \end{column}
    \begin{column}{0.65\linewidth}
    \begin{center}
    \includegraphics[max width=0.95\textwidth,
        max height=0.54000\textheight]{{Images/aardwolf}.jpg}
    \end{center}
    \end{column}
    \end{columns}
}
\end{frame}
\begin{frame}[t]{More Plants and Animals, Answer 7}
% \vspace{0.5em}
\begin{block}{Question}
What are the two types of vascular tissue responsible for transporting water and nutrients in plants?
\end{block}

\visible<2->{
    \begin{columns}[T,totalwidth=\linewidth]
    \begin{column}{0.32\linewidth}
    \begin{block}{Answer}
    Xylem and phloem
    \end{block}
    \end{column}
    \begin{column}{0.65\linewidth}
    \begin{center}
    \includegraphics[max width=0.95\textwidth,
        max height=0.50000\textheight]{{Images/xylemphloem}.png}
    \end{center}
    \end{column}
    \end{columns}
}
\end{frame}
\begin{frame}[t]{More Plants and Animals, Answer 8}
% \vspace{0.5em}
\begin{block}{Question}
What is the specific term for the chick of a hawk or falcon?
\end{block}

\visible<2->{
    \begin{columns}[T,totalwidth=\linewidth]
    \begin{column}{0.32\linewidth}
    \begin{block}{Answer}
    Eyas (pl.\ eyasses)
    \end{block}
    \end{column}
    \begin{column}{0.65\linewidth}
    \begin{center}
    \includegraphics[max width=0.95\textwidth,
        max height=0.54000\textheight]{{Images/eyas}.jpg}
    \end{center}
    \end{column}
    \end{columns}
}
\end{frame}
\begin{frame}[t]{More Plants and Animals, Answer 9}
% \vspace{0.5em}
\begin{block}{Question}
What polymer is the primary component of the exoskeletons of insects and other arthropods?
\end{block}

\visible<2->{
    \begin{columns}[T,totalwidth=\linewidth]
    \begin{column}{0.32\linewidth}
    \begin{block}{Answer}
    Chitin
    \end{block}
    \end{column}
    \begin{column}{0.65\linewidth}
    \begin{center}
    \includegraphics[max width=0.95\textwidth,
        max height=0.54000\textheight]{{Images/chitin}.png}
    \end{center}
    \end{column}
    \end{columns}
}
\end{frame}
\begin{frame}[t]{More Plants and Animals, Answer 10}
% \vspace{0.5em}
\begin{block}{Question}
Snails, octopuses, and oysters all belong to which phylum?
\end{block}

\visible<2->{
    \begin{columns}[T,totalwidth=\linewidth]
    \begin{column}{0.32\linewidth}
    \begin{block}{Answer}
    Mollusca/mollusks
    \end{block}
    \end{column}
    \begin{column}{0.65\linewidth}
    \begin{center}
    \includegraphics[max width=0.95\textwidth,
        max height=0.54000\textheight]{{Images/mollusca}.jpg}
    \end{center}
    \end{column}
    \end{columns}
}
\end{frame}
\def\thisSectionName{Geography}
\section{Round 6}
\subsection*{Q1}
\begin{frame}[t]{Geography, Question 1}
% \vspace{0.5em}
\begin{columns}[T,totalwidth=\linewidth]
\begin{column}{0.32\linewidth}
\begin{block}{Question}
What is the name of the group of islands shown in this image?
\end{block}
\end{column}
\begin{column}{0.65\linewidth}
\begin{center}
\includegraphics[max width=0.95\textwidth,max height=0.7\textheight]{{Images/galapagos}.png}
\end{center}
\end{column}
\end{columns}
\end{frame}
\subsection*{Q2}
\begin{frame}[t]{Geography, Question 2}
% \vspace{0.5em}
\begin{columns}[T,totalwidth=\linewidth]
\begin{column}{0.32\linewidth}
\begin{block}{Question}
Which mountain is pictured here?
\end{block}
\end{column}
\begin{column}{0.65\linewidth}
\begin{center}
\includegraphics[max width=0.95\textwidth,max height=0.7\textheight]{{Images/materhorn}.jpg}
\end{center}
\end{column}
\end{columns}
\end{frame}
\subsection*{Q3}
\begin{frame}[t]{Geography, Question 3}
% \vspace{0.5em}
\begin{block}{Question}
A ``doubly-landlocked country'' is a landlocked country that is entirely surrounded by other landlocked countries. What are the only two doubly-landlocked countries in the world?
\end{block}
\end{frame}
\subsection*{Q4}
\begin{frame}[t]{Geography, Question 4}
% \vspace{0.5em}
\begin{block}{Question}
By area, what is the largest country in Africa?
\end{block}
\end{frame}
\subsection*{Q5}
\begin{frame}[t]{Geography, Question 5}
% \vspace{0.5em}
\begin{block}{Question}
Everest is the highest peak on Earth, but another mountain is the tallest in the world when measuring the vertical distance from its (underwater) base to its peak. Which mountain is it?
\end{block}
\end{frame}
\subsection*{Q6}
\begin{frame}[t]{Geography, Question 6}
% \vspace{0.5em}
\begin{block}{Question}
Which Great Lake is the smallest by surface area?
\end{block}
\end{frame}
\subsection*{Q7}
\begin{frame}[t]{Geography, Question 7}
% \vspace{0.5em}
\begin{block}{Question}
To the nearest 1,000 miles, what is the circumference of the Earth at the equator?
\end{block}
\end{frame}
\subsection*{Q8}
\begin{frame}[t]{Geography, Question 8}
% \vspace{0.5em}
\begin{block}{Question}
The navigable portion of the Hudson ``River'' is actually not a river at all, but is one of these bodies of water.
\end{block}
\end{frame}
\subsection*{Q9}
\begin{frame}[t]{Geography, Question 9}
% \vspace{0.5em}
\begin{block}{Question}
What is the deepest lake in the U.S.?
\end{block}
\end{frame}
\subsection*{Q10}
\begin{frame}[t]{Geography, Question 10}
% \vspace{0.5em}
\begin{block}{Question}
The Nile and the Amazon are the two longest rivers in the world. Which river is third longest?
\end{block}
\end{frame}
\subsection{Answers}
\begin{frame}[t]{Geography, Answer 1}
% \vspace{0.5em}
\begin{columns}[T,totalwidth=\linewidth]
\begin{column}{0.32\linewidth}
\begin{block}{Question}
What is the name of the group of islands shown in this image?
\end{block}
\visible<2->{
    \begin{block}{Answer}
    The Galapagos Islands
    \end{block}
}
\end{column}
\begin{column}{0.65\linewidth}
\begin{center}
\includegraphics[max width=0.95\textwidth,max height=0.7\textheight]{{Images/galapagos}.png}
\end{center}
\end{column}
\end{columns}
\end{frame}
\begin{frame}[t]{Geography, Answer 2}
% \vspace{0.5em}
\begin{columns}[T,totalwidth=\linewidth]
\begin{column}{0.32\linewidth}
\begin{block}{Question}
Which mountain is pictured here?
\end{block}
\visible<2->{
    \begin{block}{Answer}
    The Matterhorn
    \end{block}
}
\end{column}
\begin{column}{0.65\linewidth}
\begin{center}
\includegraphics[max width=0.95\textwidth,max height=0.7\textheight]{{Images/materhorn}.jpg}
\end{center}
\end{column}
\end{columns}
\end{frame}
\begin{frame}[t]{Geography, Answer 3}
% \vspace{0.5em}
\begin{block}{Question}
A ``doubly-landlocked country'' is a landlocked country that is entirely surrounded by other landlocked countries. What are the only two doubly-landlocked countries in the world?
\end{block}

\visible<2->{
    \begin{columns}[T,totalwidth=\linewidth]
    \begin{column}{0.32\linewidth}
    \begin{block}{Answer}
    Lichtenstein and Uzbekistan
    \end{block}
    \end{column}
    \begin{column}{0.65\linewidth}
    \begin{center}
    \includegraphics[max width=0.95\textwidth,
        max height=0.46000\textheight]{{Images/dll}.jpg}
    \end{center}
    \end{column}
    \end{columns}
}
\end{frame}
\begin{frame}[t]{Geography, Answer 4}
% \vspace{0.5em}
\begin{block}{Question}
By area, what is the largest country in Africa?
\end{block}

\visible<2->{
    \begin{columns}[T,totalwidth=\linewidth]
    \begin{column}{0.32\linewidth}
    \begin{block}{Answer}
    Algeria (919,595 sq miles)
    \end{block}
    \end{column}
    \begin{column}{0.65\linewidth}
    \begin{center}
    \includegraphics[max width=0.95\textwidth,
        max height=0.58000\textheight]{{Images/algeriamap}.png}
    \end{center}
    \end{column}
    \end{columns}
}
\end{frame}
\begin{frame}[t]{Geography, Answer 5}
% \vspace{0.5em}
\begin{block}{Question}
Everest is the highest peak on Earth, but another mountain is the tallest in the world when measuring the vertical distance from its (underwater) base to its peak. Which mountain is it?
\end{block}

\visible<2->{
    \begin{columns}[T,totalwidth=\linewidth]
    \begin{column}{0.32\linewidth}
    \begin{block}{Answer}
    Mauna Kea (33,500 ft; peak is 13,803 ft above sea level)
    \end{block}
    \end{column}
    \begin{column}{0.65\linewidth}
    \begin{center}
    \includegraphics[max width=0.95\textwidth,
        max height=0.46000\textheight]{{Images/maunakea}.jpg}
    \end{center}
    \end{column}
    \end{columns}
}
\end{frame}
\begin{frame}[t]{Geography, Answer 6}
% \vspace{0.5em}
\begin{block}{Question}
Which Great Lake is the smallest by surface area?
\end{block}

\visible<2->{
    \begin{columns}[T,totalwidth=\linewidth]
    \begin{column}{0.32\linewidth}
    \begin{block}{Answer}
    Lake Ontario
    \end{block}
    \end{column}
    \begin{column}{0.65\linewidth}
    \begin{center}
    \includegraphics[max width=0.95\textwidth,
        max height=0.58000\textheight]{{Images/greatlakes}.png}
    \end{center}
    \end{column}
    \end{columns}
}
\end{frame}
\begin{frame}[t]{Geography, Answer 7}
% \vspace{0.5em}
\begin{block}{Question}
To the nearest 1,000 miles, what is the circumference of the Earth at the equator?
\end{block}

\visible<2->{
    \begin{columns}[T,totalwidth=\linewidth]
    \begin{column}{0.32\linewidth}
    \begin{block}{Answer}
    25,000 mi (24,901 exactly)
    \end{block}
    \end{column}
    \begin{column}{0.65\linewidth}
    \begin{center}
    \includegraphics[max width=0.95\textwidth,
        max height=0.54000\textheight]{{Images/earthcircumference}.png}
    \end{center}
    \end{column}
    \end{columns}
}
\end{frame}
\begin{frame}[t]{Geography, Answer 8}
% \vspace{0.5em}
\begin{block}{Question}
The navigable portion of the Hudson ``River'' is actually not a river at all, but is one of these bodies of water.
\end{block}

\visible<2->{
    \begin{columns}[T,totalwidth=\linewidth]
    \begin{column}{0.32\linewidth}
    \begin{block}{Answer}
    A (tidal) estuary
    \end{block}
    \end{column}
    \begin{column}{0.65\linewidth}
    \begin{center}
    \includegraphics[max width=0.95\textwidth,
        max height=0.50000\textheight]{{Images/estuary}.jpg}
    \end{center}
    \end{column}
    \end{columns}
}
\end{frame}
\begin{frame}[t]{Geography, Answer 9}
% \vspace{0.5em}
\begin{block}{Question}
What is the deepest lake in the U.S.?
\end{block}

\visible<2->{
    \begin{columns}[T,totalwidth=\linewidth]
    \begin{column}{0.32\linewidth}
    \begin{block}{Answer}
    Crater Lake (Oregon, 1,949 ft)
    \end{block}
    \end{column}
    \begin{column}{0.65\linewidth}
    \begin{center}
    \includegraphics[max width=0.95\textwidth,
        max height=0.58000\textheight]{{Images/craterlake}.jpg}
    \end{center}
    \end{column}
    \end{columns}
}
\end{frame}
\begin{frame}[t]{Geography, Answer 10}
% \vspace{0.5em}
\begin{block}{Question}
The Nile and the Amazon are the two longest rivers in the world. Which river is third longest?
\end{block}

\visible<2->{
    \begin{columns}[T,totalwidth=\linewidth]
    \begin{column}{0.32\linewidth}
    \begin{block}{Answer}
    The Yangtze
    \end{block}
    \end{column}
    \begin{column}{0.65\linewidth}
    \begin{center}
    \includegraphics[max width=0.95\textwidth,
        max height=0.54000\textheight]{{Images/yangtze}.jpg}
    \end{center}
    \end{column}
    \end{columns}
}
\end{frame}
\def\thisSectionName{Real name/Stage name}
\section{Round 7}
\subsection*{Q1}
\begin{frame}[t]{Real name/Stage name, Question 1}
% \vspace{0.5em}
\begin{block}{Question}
The singer born Onika Tanya Maraj-Petty is better known by what stage name?
\end{block}
\end{frame}
\subsection*{Q2}
\begin{frame}[t]{Real name/Stage name, Question 2}
% \vspace{0.5em}
\begin{block}{Question}
When choosing her stage name, the actress born Margaret Mary Emily Hyra chose a last name that's nearly an anagram of ``Hyra''. What is her stage name?
\end{block}
\end{frame}
\subsection*{Q3}
\begin{frame}[t]{Real name/Stage name, Question 3}
% \vspace{0.5em}
\begin{block}{Question}
The actor born Thomas Mapother IV is better known by what stage name?
\end{block}
\end{frame}
\subsection*{Q4}
\begin{frame}[t]{Real name/Stage name, Question 4}
% \vspace{0.5em}
\begin{block}{Question}
Jon Bon Jovi, who is partly of  Italian heritage, didn't have to change his birth name too much to arrive at his stage name. What was his original last name? (Spelling counts on this one.)
\end{block}
\end{frame}
\subsection*{Q5}
\begin{frame}[t]{Real name/Stage name, Question 5}
% \vspace{0.5em}
\begin{block}{Question}
Lots of people mistakenly think this actor's stage name's first name is ``William'' --- and they're not far off, because that \emph{is} his birth name's first name. What stage name is this actor better known by?
\end{block}
\end{frame}
\subsection*{Q6}
\begin{frame}[t]{Real name/Stage name, Question 6}
% \vspace{0.5em}
\begin{block}{Question}
It's debatable whether his whole life is a lie, but the actor born Jay Scott Greenspan does not go by his birth name. What stage name is he better known by?
\end{block}
\end{frame}
\subsection*{Q7}
\begin{frame}[t]{Real name/Stage name, Question 7}
% \vspace{0.5em}
\begin{block}{Question}
Name the chanteuse born Freda Josephine McDonald.
\end{block}
\end{frame}
\subsection*{Q8}
\begin{frame}[t]{Real name/Stage name, Question 8}
% \vspace{0.5em}
\begin{block}{Question}
The stage name of this five-time Grammy Award winner is just the first two names of her legal birth name; she dropped the last three, ``Pirate Baird O'Connell''. Who's the singer?
\end{block}
\end{frame}
\subsection*{Q9}
\begin{frame}[t]{Real name/Stage name, Question 9}
% \vspace{0.5em}
\begin{block}{Question}
Mark Sinclair is the birth name of which action movie star?
\end{block}
\end{frame}
\subsection*{Q10}
\begin{frame}[t]{Real name/Stage name, Question 10}
% \vspace{0.5em}
\begin{block}{Question}
What were the stage names of Ethel Mae Blythe, John Sidney Blythe and Lionel Herbert Blythe? (You must get all three.)
\end{block}
\end{frame}
\subsection{Answers}
\begin{frame}[t]{Real name/Stage name, Answer 1}
% \vspace{0.5em}
\begin{block}{Question}
The singer born Onika Tanya Maraj-Petty is better known by what stage name?
\end{block}

\visible<2->{
    \begin{columns}[T,totalwidth=\linewidth]
    \begin{column}{0.32\linewidth}
    \begin{block}{Answer}
    Nicki Minaj
    \end{block}
    \end{column}
    \begin{column}{0.65\linewidth}
    \begin{center}
    \includegraphics[max width=0.95\textwidth,
        max height=0.54000\textheight]{{Images/nickiminaj}.jpg}
    \end{center}
    \end{column}
    \end{columns}
}
\end{frame}
\begin{frame}[t]{Real name/Stage name, Answer 2}
% \vspace{0.5em}
\begin{block}{Question}
When choosing her stage name, the actress born Margaret Mary Emily Hyra chose a last name that's nearly an anagram of ``Hyra''. What is her stage name?
\end{block}

\visible<2->{
    \begin{columns}[T,totalwidth=\linewidth]
    \begin{column}{0.32\linewidth}
    \begin{block}{Answer}
    Meg Ryan
    \end{block}
    \end{column}
    \begin{column}{0.65\linewidth}
    \begin{center}
    \includegraphics[max width=0.95\textwidth,
        max height=0.46000\textheight]{{Images/megryan}.jpg}
    \end{center}
    \end{column}
    \end{columns}
}
\end{frame}
\begin{frame}[t]{Real name/Stage name, Answer 3}
% \vspace{0.5em}
\begin{block}{Question}
The actor born Thomas Mapother IV is better known by what stage name?
\end{block}

\visible<2->{
    \begin{columns}[T,totalwidth=\linewidth]
    \begin{column}{0.32\linewidth}
    \begin{block}{Answer}
    Tom Cruise
    \end{block}
    \end{column}
    \begin{column}{0.65\linewidth}
    \begin{center}
    \includegraphics[max width=0.95\textwidth,
        max height=0.54000\textheight]{{Images/tomcruise}.jpg}
    \end{center}
    \end{column}
    \end{columns}
}
\end{frame}
\begin{frame}[t]{Real name/Stage name, Answer 4}
% \vspace{0.5em}
\begin{block}{Question}
Jon Bon Jovi, who is partly of  Italian heritage, didn't have to change his birth name too much to arrive at his stage name. What was his original last name? (Spelling counts on this one.)
\end{block}

\visible<2->{
    \begin{columns}[T,totalwidth=\linewidth]
    \begin{column}{0.32\linewidth}
    \begin{block}{Answer}
    Bongiovi (full name John Francis Bongiovi, Jr.)
    \end{block}
    \end{column}
    \begin{column}{0.65\linewidth}
    \begin{center}
    \includegraphics[max width=0.95\textwidth,
        max height=0.46000\textheight]{{Images/bonjovi}.jpeg}
    \end{center}
    \end{column}
    \end{columns}
}
\end{frame}
\begin{frame}[t]{Real name/Stage name, Answer 5}
% \vspace{0.5em}
\begin{block}{Question}
Lots of people mistakenly think this actor's stage name's first name is ``William'' --- and they're not far off, because that \emph{is} his birth name's first name. What stage name is this actor better known by?
\end{block}

\visible<2->{
    \begin{columns}[T,totalwidth=\linewidth]
    \begin{column}{0.32\linewidth}
    \begin{block}{Answer}
    Willem Dafoe
    \end{block}
    \end{column}
    \begin{column}{0.65\linewidth}
    \begin{center}
    \includegraphics[max width=0.95\textwidth,
        max height=0.42000\textheight]{{Images/dafoe}.jpg}
    \end{center}
    \end{column}
    \end{columns}
}
\end{frame}
\begin{frame}[t]{Real name/Stage name, Answer 6}
% \vspace{0.5em}
\begin{block}{Question}
It's debatable whether his whole life is a lie, but the actor born Jay Scott Greenspan does not go by his birth name. What stage name is he better known by?
\end{block}

\visible<2->{
    \begin{columns}[T,totalwidth=\linewidth]
    \begin{column}{0.32\linewidth}
    \begin{block}{Answer}
    Jason Alexander
    \end{block}
    \end{column}
    \begin{column}{0.65\linewidth}
    \begin{center}
    \includegraphics[max width=0.95\textwidth,
        max height=0.46000\textheight]{{Images/costanza}.jpg}
    \end{center}
    \end{column}
    \end{columns}
}
\end{frame}
\begin{frame}[t]{Real name/Stage name, Answer 7}
% \vspace{0.5em}
\begin{block}{Question}
Name the chanteuse born Freda Josephine McDonald.
\end{block}

\visible<2->{
    \begin{columns}[T,totalwidth=\linewidth]
    \begin{column}{0.32\linewidth}
    \begin{block}{Answer}
    Josephine Baker
    \end{block}
    \end{column}
    \begin{column}{0.65\linewidth}
    \begin{center}
    \includegraphics[max width=0.95\textwidth,
        max height=0.58000\textheight]{{Images/josephinebaker}.jpg}
    \end{center}
    \end{column}
    \end{columns}
}
\end{frame}
\begin{frame}[t]{Real name/Stage name, Answer 8}
% \vspace{0.5em}
\begin{block}{Question}
The stage name of this five-time Grammy Award winner is just the first two names of her legal birth name; she dropped the last three, ``Pirate Baird O'Connell''. Who's the singer?
\end{block}

\visible<2->{
    \begin{columns}[T,totalwidth=\linewidth]
    \begin{column}{0.32\linewidth}
    \begin{block}{Answer}
    Billie Eilish
    \end{block}
    \end{column}
    \begin{column}{0.65\linewidth}
    \begin{center}
    \includegraphics[max width=0.95\textwidth,
        max height=0.46000\textheight]{{Images/billieeilish}.jpg}
    \end{center}
    \end{column}
    \end{columns}
}
\end{frame}
\begin{frame}[t]{Real name/Stage name, Answer 9}
% \vspace{0.5em}
\begin{block}{Question}
Mark Sinclair is the birth name of which action movie star?
\end{block}

\visible<2->{
    \begin{columns}[T,totalwidth=\linewidth]
    \begin{column}{0.32\linewidth}
    \begin{block}{Answer}
    Vin Diesel
    \end{block}
    \end{column}
    \begin{column}{0.65\linewidth}
    \begin{center}
    \includegraphics[max width=0.95\textwidth,
        max height=0.54000\textheight]{{Images/vindiesel}.jpeg}
    \end{center}
    \end{column}
    \end{columns}
}
\end{frame}
\begin{frame}[t]{Real name/Stage name, Answer 10}
% \vspace{0.5em}
\begin{block}{Question}
What were the stage names of Ethel Mae Blythe, John Sidney Blythe and Lionel Herbert Blythe? (You must get all three.)
\end{block}

\visible<2->{
    \begin{columns}[T,totalwidth=\linewidth]
    \begin{column}{0.32\linewidth}
    \begin{block}{Answer}
    Ethel, John and Lionel Barrymore
    \end{block}
    \end{column}
    \begin{column}{0.65\linewidth}
    \begin{center}
    \includegraphics[max width=0.95\textwidth,
        max height=0.50000\textheight]{{Images/barrymore}.jpg}
    \end{center}
    \end{column}
    \end{columns}
}
\end{frame}
\def\thisSectionName{What are they saying about me?}
\section{Round 8}
\subsection*{Q1}
\begin{frame}[t]{What are they saying about me?, Question 1}
% \vspace{0.5em}
\begin{block}{Question}
Who was novelist and critic Mary McCarthy talking about when she said, ``Every word she writes is a lie, including `and' and `the'\,''?
\end{block}
\end{frame}
\subsection*{Q2}
\begin{frame}[t]{What are they saying about me?, Question 2}
% \vspace{0.5em}
\begin{block}{Question}
Who said, ``I knew Doris Day before she was a virgin''?
\end{block}
\end{frame}
\subsection*{Q3}
\begin{frame}[t]{What are they saying about me?, Question 3}
% \vspace{0.5em}
\begin{block}{Question}
Who was Vladimir Nabokov talking about when he said, ``I read him for the first time in the early forties, something about bells, balls and bulls, and loathed it.''
\end{block}
\end{frame}
\subsection*{Q4}
\begin{frame}[t]{What are they saying about me?, Question 4}
% \vspace{0.5em}
\begin{block}{Question}
``Her forehead looks like a f**king flat screen TV.'' Which actress was Sharon Osbourne talking about?
\end{block}
\end{frame}
\subsection*{Q5}
\begin{frame}[t]{What are they saying about me?, Question 5}
% \vspace{0.5em}
\begin{block}{Question}
Etta James said of a fellow singer who had sung Ms.\ James' signature song ``At Last'' at Pres. Obama's inauguration festivities, ``I tell you that woman he had singing for him, singing my song, she gonna get her ass whupped.'' Who was Ms.\ James talking about?  
\end{block}
\end{frame}
\subsection*{Q6}
\begin{frame}[t]{What are they saying about me?, Question 6}
% \vspace{0.5em}
\begin{block}{Question}
Who was a 19\textsuperscript{th} century New York Times reporter talking about when he wrote, ``Everything which made [him] the loved and honored man he was, it is in the power of the humblest American boy to imitate.''
\end{block}
\end{frame}
\subsection*{Q7}
\begin{frame}[t]{What are they saying about me?, Question 7}
% \vspace{0.5em}
\begin{block}{Question}
``None of these people\ldots{} have anything interesting to say, and none of them can write, not even Mr.\ Kerouac. It's not writing it's typing.'' Who was the writer who was so down on the Beats?
\end{block}
\end{frame}
\subsection*{Q8}
\begin{frame}[t]{What are they saying about me?, Question 8}
% \vspace{0.5em}
\begin{block}{Question}
When Yankee Reggie Jackson remarked to a reporter that he had an IQ of 160, another ballplayer who overheard him said, ``Out of what, a thousand?'' Who said it?
\end{block}
\end{frame}
\subsection*{Q9}
\begin{frame}[t]{What are they saying about me?, Question 9}
% \vspace{0.5em}
\begin{block}{Question}
Which novelist known for his acerbic wit said of Truman Capote, ``He's a full-fledged housewife from Kansas with all the prejudices''?
\end{block}
\end{frame}
\subsection*{Q10}
\begin{frame}[t]{What are they saying about me?, Question 10}
% \vspace{0.5em}
\begin{block}{Question}
What Nobel Prize-winning author called Mark Twain, ``[A] hack writer who would not have been considered fourth rate in Europe\ldots{} with sufficient local color to intrigue the superficial and the lazy''?
\end{block}
\end{frame}
\subsection{Answers}
\begin{frame}[t]{What are they saying about me?, Answer 1}
% \vspace{0.5em}
\begin{block}{Question}
Who was novelist and critic Mary McCarthy talking about when she said, ``Every word she writes is a lie, including `and' and `the'\,''?
\end{block}

\visible<2->{
    \begin{columns}[T,totalwidth=\linewidth]
    \begin{column}{0.32\linewidth}
    \begin{block}{Answer}
    Lillian Hellman
    \end{block}
    \end{column}
    \begin{column}{0.65\linewidth}
    \begin{center}
    \includegraphics[max width=0.95\textwidth,
        max height=0.50000\textheight]{{Images/hellman}.jpg}
    \end{center}
    \end{column}
    \end{columns}
}
\end{frame}
\begin{frame}[t]{What are they saying about me?, Answer 2}
% \vspace{0.5em}
\begin{block}{Question}
Who said, ``I knew Doris Day before she was a virgin''?
\end{block}

\visible<2->{
    \begin{columns}[T,totalwidth=\linewidth]
    \begin{column}{0.32\linewidth}
    \begin{block}{Answer}
    Oscar Levant
    \end{block}
    \end{column}
    \begin{column}{0.65\linewidth}
    \begin{center}
    \includegraphics[max width=0.95\textwidth,
        max height=0.54000\textheight]{{Images/levant}.jpeg}
    \end{center}
    \end{column}
    \end{columns}
}
\end{frame}
\begin{frame}[t]{What are they saying about me?, Answer 3}
% \vspace{0.5em}
\begin{block}{Question}
Who was Vladimir Nabokov talking about when he said, ``I read him for the first time in the early forties, something about bells, balls and bulls, and loathed it.''
\end{block}

\visible<2->{
    \begin{columns}[T,totalwidth=\linewidth]
    \begin{column}{0.32\linewidth}
    \begin{block}{Answer}
    Ernest Hemingway
    \end{block}
    \end{column}
    \begin{column}{0.65\linewidth}
    \begin{center}
    \includegraphics[max width=0.95\textwidth,
        max height=0.46000\textheight]{{Images/hemingway}.jpg}
    \end{center}
    \end{column}
    \end{columns}
}
\end{frame}
\begin{frame}[t]{What are they saying about me?, Answer 4}
% \vspace{0.5em}
\begin{block}{Question}
``Her forehead looks like a f**king flat screen TV.'' Which actress was Sharon Osbourne talking about?
\end{block}

\visible<2->{
    \begin{columns}[T,totalwidth=\linewidth]
    \begin{column}{0.32\linewidth}
    \begin{block}{Answer}
    Nicole Kidman
    \end{block}
    \end{column}
    \begin{column}{0.65\linewidth}
    \begin{center}
    \includegraphics[max width=0.95\textwidth,
        max height=0.50000\textheight]{{Images/kidman}.jpeg}
    \end{center}
    \end{column}
    \end{columns}
}
\end{frame}
\begin{frame}[t]{What are they saying about me?, Answer 5}
% \vspace{0.5em}
\begin{block}{Question}
Etta James said of a fellow singer who had sung Ms.\ James' signature song ``At Last'' at Pres. Obama's inauguration festivities, ``I tell you that woman he had singing for him, singing my song, she gonna get her ass whupped.'' Who was Ms.\ James talking about?  
\end{block}

\visible<2->{
    \begin{columns}[T,totalwidth=\linewidth]
    \begin{column}{0.32\linewidth}
    \begin{block}{Answer}
    Beyoncé
    \end{block}
    \end{column}
    \begin{column}{0.65\linewidth}
    \begin{center}
    \includegraphics[max width=0.95\textwidth,
        max height=0.38000\textheight]{{Images/beyonceetta}.jpeg}
    \end{center}
    \end{column}
    \end{columns}
}
\end{frame}
\begin{frame}[t]{What are they saying about me?, Answer 6}
% \vspace{0.5em}
\begin{block}{Question}
Who was a 19\textsuperscript{th} century New York Times reporter talking about when he wrote, ``Everything which made [him] the loved and honored man he was, it is in the power of the humblest American boy to imitate.''
\end{block}

\visible<2->{
    \begin{columns}[T,totalwidth=\linewidth]
    \begin{column}{0.32\linewidth}
    \begin{block}{Answer}
    Abraham Lincoln
    \end{block}
    \end{column}
    \begin{column}{0.65\linewidth}
    \begin{center}
    \includegraphics[max width=0.95\textwidth,
        max height=0.42000\textheight]{{Images/lincoln}.jpg}
    \end{center}
    \end{column}
    \end{columns}
}
\end{frame}
\begin{frame}[t]{What are they saying about me?, Answer 7}
% \vspace{0.5em}
\begin{block}{Question}
``None of these people\ldots{} have anything interesting to say, and none of them can write, not even Mr.\ Kerouac. It's not writing it's typing.'' Who was the writer who was so down on the Beats?
\end{block}

\visible<2->{
    \begin{columns}[T,totalwidth=\linewidth]
    \begin{column}{0.32\linewidth}
    \begin{block}{Answer}
    Truman Capote
    \end{block}
    \end{column}
    \begin{column}{0.65\linewidth}
    \begin{center}
    \includegraphics[max width=0.95\textwidth,
        max height=0.46000\textheight]{{Images/capote}.jpg}
    \end{center}
    \end{column}
    \end{columns}
}
\end{frame}
\begin{frame}[t]{What are they saying about me?, Answer 8}
% \vspace{0.5em}
\begin{block}{Question}
When Yankee Reggie Jackson remarked to a reporter that he had an IQ of 160, another ballplayer who overheard him said, ``Out of what, a thousand?'' Who said it?
\end{block}

\visible<2->{
    \begin{columns}[T,totalwidth=\linewidth]
    \begin{column}{0.32\linewidth}
    \begin{block}{Answer}
    Mickey Rivers
    \end{block}
    \end{column}
    \begin{column}{0.65\linewidth}
    \begin{center}
    \includegraphics[max width=0.95\textwidth,
        max height=0.46000\textheight]{{Images/mickeyrivers}.jpeg}
    \end{center}
    \end{column}
    \end{columns}
}
\end{frame}
\begin{frame}[t]{What are they saying about me?, Answer 9}
% \vspace{0.5em}
\begin{block}{Question}
Which novelist known for his acerbic wit said of Truman Capote, ``He's a full-fledged housewife from Kansas with all the prejudices''?
\end{block}

\visible<2->{
    \begin{columns}[T,totalwidth=\linewidth]
    \begin{column}{0.32\linewidth}
    \begin{block}{Answer}
    Gore Vidal
    \end{block}
    \end{column}
    \begin{column}{0.65\linewidth}
    \begin{center}
    \includegraphics[max width=0.95\textwidth,
        max height=0.50000\textheight]{{Images/vidal}.jpg}
    \end{center}
    \end{column}
    \end{columns}
}
\end{frame}
\begin{frame}[t]{What are they saying about me?, Answer 10}
% \vspace{0.5em}
\begin{block}{Question}
What Nobel Prize-winning author called Mark Twain, ``[A] hack writer who would not have been considered fourth rate in Europe\ldots{} with sufficient local color to intrigue the superficial and the lazy''?
\end{block}

\visible<2->{
    \begin{columns}[T,totalwidth=\linewidth]
    \begin{column}{0.32\linewidth}
    \begin{block}{Answer}
    William Faulkner
    \end{block}
    \end{column}
    \begin{column}{0.65\linewidth}
    \begin{center}
    \includegraphics[max width=0.95\textwidth,
        max height=0.42000\textheight]{{Images/faulkner}.jpg}
    \end{center}
    \end{column}
    \end{columns}
}
\end{frame}
\def\thisSectionName{Ancient Civilizations}
\section{Round 9}
\subsection*{Q1}
\begin{frame}[t]{Ancient Civilizations, Question 1}
% \vspace{0.5em}
\begin{block}{Question}
Which river, the name of which is now part of a phrase meaning ``past the point of no return'', did Julius Caesar cross with his army in contravention of the tradition of republican Rome?
\end{block}
\end{frame}
\subsection*{Q2}
\begin{frame}[t]{Ancient Civilizations, Question 2}
% \vspace{0.5em}
\begin{block}{Question}
Which war ended the Golden Age of Athens?
\end{block}
\end{frame}
\subsection*{Q3}
\begin{frame}[t]{Ancient Civilizations, Question 3}
% \vspace{0.5em}
\begin{block}{Question}
To the nearest 500 years, when did the Bronze Age begin?
\end{block}
\end{frame}
\subsection*{Q4}
\begin{frame}[t]{Ancient Civilizations, Question 4}
% \vspace{0.5em}
\begin{columns}[T,totalwidth=\linewidth]
\begin{column}{0.32\linewidth}
\begin{block}{Question}
What is the name of the writing system, pictured here, that was used in Ancient Mesopotamia and is named for its wedge-shaped characters?
\end{block}
\end{column}
\begin{column}{0.65\linewidth}
\begin{center}
\includegraphics[max width=0.95\textwidth,max height=0.7\textheight]{{Images/cuneiform}.jpg}
\end{center}
\end{column}
\end{columns}
\end{frame}
\subsection*{Q5}
\begin{frame}[t]{Ancient Civilizations, Question 5}
% \vspace{0.5em}
\begin{block}{Question}
The Roman Colosseum is famous for its gladiator fights, but what aquatic events were also held there?
\end{block}
\end{frame}
\subsection*{Q6}
\begin{frame}[t]{Ancient Civilizations, Question 6}
% \vspace{0.5em}
\begin{block}{Question}
The earliest know instances of Chinese writing are carvings in animal bones that were used for divination in Ancient China. What two-word phrase are these bones known by?
\end{block}
\end{frame}
\subsection*{Q7}
\begin{frame}[t]{Ancient Civilizations, Question 7}
% \vspace{0.5em}
\begin{block}{Question}
Ancient Babylon was situated near what present-day world capital?
\end{block}
\end{frame}
\subsection*{Q8}
\begin{frame}[t]{Ancient Civilizations, Question 8}
% \vspace{0.5em}
\begin{columns}[T,totalwidth=\linewidth]
\begin{column}{0.32\linewidth}
\begin{block}{Question}
Which Egyptian god, depicted here as a human with the head of an ibis, was the god of wisdom, science, and writing?
\end{block}
\end{column}
\begin{column}{0.65\linewidth}
\begin{center}
\includegraphics[max width=0.95\textwidth,max height=0.7\textheight]{{Images/thoth}.png}
\end{center}
\end{column}
\end{columns}
\end{frame}
\subsection*{Q9}
\begin{frame}[t]{Ancient Civilizations, Question 9}
% \vspace{0.5em}
\begin{block}{Question}
To which language family, which began diverging into different languages circa  2000 BC, did the ancient languages Latin, Greek, Thracian, Hittite, and Sanskrit all belong?
\end{block}
\end{frame}
\subsection*{Q10}
\begin{frame}[t]{Ancient Civilizations, Question 10}
% \vspace{0.5em}
\begin{block}{Question}
From which ancient civilization did Hannibal, famous for leading his army and his elephants into Italy across the the Alps, hail?
\end{block}
\end{frame}
\subsection{Answers}
\begin{frame}[t]{Ancient Civilizations, Answer 1}
% \vspace{0.5em}
\begin{block}{Question}
Which river, the name of which is now part of a phrase meaning ``past the point of no return'', did Julius Caesar cross with his army in contravention of the tradition of republican Rome?
\end{block}

\visible<2->{
    \begin{columns}[T,totalwidth=\linewidth]
    \begin{column}{0.32\linewidth}
    \begin{block}{Answer}
    The Rubicon
    \end{block}
    \end{column}
    \begin{column}{0.65\linewidth}
    \begin{center}
    \includegraphics[max width=0.95\textwidth,
        max height=0.46000\textheight]{{Images/rubicon}.jpg}
    \end{center}
    \end{column}
    \end{columns}
}
\end{frame}
\begin{frame}[t]{Ancient Civilizations, Answer 2}
% \vspace{0.5em}
\begin{block}{Question}
Which war ended the Golden Age of Athens?
\end{block}

\visible<2->{
    \begin{columns}[T,totalwidth=\linewidth]
    \begin{column}{0.32\linewidth}
    \begin{block}{Answer}
    The Peloponnesian War (431--404 BC)
    \end{block}
    \end{column}
    \begin{column}{0.65\linewidth}
    \begin{center}
    \includegraphics[max width=0.95\textwidth,
        max height=0.58000\textheight]{{Images/peloponnesian}.jpg}
    \end{center}
    \end{column}
    \end{columns}
}
\end{frame}
\begin{frame}[t]{Ancient Civilizations, Answer 3}
% \vspace{0.5em}
\begin{block}{Question}
To the nearest 500 years, when did the Bronze Age begin?
\end{block}

\visible<2->{
    \begin{columns}[T,totalwidth=\linewidth]
    \begin{column}{0.32\linewidth}
    \begin{block}{Answer}
    3000 BC
    \end{block}
    \end{column}
    \begin{column}{0.65\linewidth}
    \begin{center}
    \includegraphics[max width=0.95\textwidth,
        max height=0.54000\textheight]{{Images/bronzeaxes}.jpg}
    \end{center}
    \end{column}
    \end{columns}
}
\end{frame}
\begin{frame}[t]{Ancient Civilizations, Answer 4}
% \vspace{0.5em}
\begin{columns}[T,totalwidth=\linewidth]
\begin{column}{0.32\linewidth}
\begin{block}{Question}
What is the name of the writing system, pictured here, that was used in Ancient Mesopotamia and is named for its wedge-shaped characters?
\end{block}
\visible<2->{
    \begin{block}{Answer}
    Cuneiform
    \end{block}
}
\end{column}
\begin{column}{0.65\linewidth}
\begin{center}
\includegraphics[max width=0.95\textwidth,max height=0.7\textheight]{{Images/cuneiform}.jpg}
\end{center}
\end{column}
\end{columns}
\end{frame}
\begin{frame}[t]{Ancient Civilizations, Answer 5}
% \vspace{0.5em}
\begin{block}{Question}
The Roman Colosseum is famous for its gladiator fights, but what aquatic events were also held there?
\end{block}

\visible<2->{
    \begin{columns}[T,totalwidth=\linewidth]
    \begin{column}{0.32\linewidth}
    \begin{block}{Answer}
    Naval battles (or \emph{naumachia}, or \emph{navalia proelia})
    \end{block}
    \end{column}
    \begin{column}{0.65\linewidth}
    \begin{center}
    \includegraphics[max width=0.95\textwidth,
        max height=0.50000\textheight]{{Images/naumaquia}.jpg}
    \end{center}
    \end{column}
    \end{columns}
}
\end{frame}
\begin{frame}[t]{Ancient Civilizations, Answer 6}
% \vspace{0.5em}
\begin{block}{Question}
The earliest know instances of Chinese writing are carvings in animal bones that were used for divination in Ancient China. What two-word phrase are these bones known by?
\end{block}

\visible<2->{
    \begin{columns}[T,totalwidth=\linewidth]
    \begin{column}{0.32\linewidth}
    \begin{block}{Answer}
    Oracle bones
    \end{block}
    \end{column}
    \begin{column}{0.65\linewidth}
    \begin{center}
    \includegraphics[max width=0.95\textwidth,
        max height=0.46000\textheight]{{Images/oraclebone}.jpg}
    \end{center}
    \end{column}
    \end{columns}
}
\end{frame}
\begin{frame}[t]{Ancient Civilizations, Answer 7}
% \vspace{0.5em}
\begin{block}{Question}
Ancient Babylon was situated near what present-day world capital?
\end{block}

\visible<2->{
    \begin{columns}[T,totalwidth=\linewidth]
    \begin{column}{0.32\linewidth}
    \begin{block}{Answer}
    Baghdad, Iraq
    \end{block}
    \end{column}
    \begin{column}{0.65\linewidth}
    \begin{center}
    \includegraphics[max width=0.95\textwidth,
        max height=0.54000\textheight]{{Images/babylon}.jpg}
    \end{center}
    \end{column}
    \end{columns}
}
\end{frame}
\begin{frame}[t]{Ancient Civilizations, Answer 8}
% \vspace{0.5em}
\begin{columns}[T,totalwidth=\linewidth]
\begin{column}{0.32\linewidth}
\begin{block}{Question}
Which Egyptian god, depicted here as a human with the head of an ibis, was the god of wisdom, science, and writing?
\end{block}
\visible<2->{
    \begin{block}{Answer}
    Thoth
    \end{block}
}
\end{column}
\begin{column}{0.65\linewidth}
\begin{center}
\includegraphics[max width=0.95\textwidth,max height=0.7\textheight]{{Images/thoth}.png}
\end{center}
\end{column}
\end{columns}
\end{frame}
\begin{frame}[t]{Ancient Civilizations, Answer 9}
% \vspace{0.5em}
\begin{block}{Question}
To which language family, which began diverging into different languages circa  2000 BC, did the ancient languages Latin, Greek, Thracian, Hittite, and Sanskrit all belong?
\end{block}

\visible<2->{
    \begin{columns}[T,totalwidth=\linewidth]
    \begin{column}{0.32\linewidth}
    \begin{block}{Answer}
    Indo-European
    \end{block}
    \end{column}
    \begin{column}{0.65\linewidth}
    \begin{center}
    \includegraphics[max width=0.95\textwidth,
        max height=0.46000\textheight]{{Images/indoeuropean}.png}
    \end{center}
    \end{column}
    \end{columns}
}
\end{frame}
\begin{frame}[t]{Ancient Civilizations, Answer 10}
% \vspace{0.5em}
\begin{block}{Question}
From which ancient civilization did Hannibal, famous for leading his army and his elephants into Italy across the the Alps, hail?
\end{block}

\visible<2->{
    \begin{columns}[T,totalwidth=\linewidth]
    \begin{column}{0.32\linewidth}
    \begin{block}{Answer}
    Carthage (present-day Tunisia)
    \end{block}
    \end{column}
    \begin{column}{0.65\linewidth}
    \begin{center}
    \includegraphics[max width=0.95\textwidth,
        max height=0.50000\textheight]{{Images/carthage}.jpg}
    \end{center}
    \end{column}
    \end{columns}
}
\end{frame}
\def\thisSectionName{TV}
\section{Round 10}
\subsection*{Q1}
\begin{frame}[t]{TV, Question 1}
% \vspace{0.5em}
\begin{block}{Question}
On \emph{The Jeffersons}, George Jefferson owned a chain of what kind of businesses?
\end{block}
\end{frame}
\subsection*{Q2}
\begin{frame}[t]{TV, Question 2}
% \vspace{0.5em}
\begin{block}{Question}
Which TV comedy followed the production of the fictional TV show \emph{The Girlie Show}?
\end{block}
\end{frame}
\subsection*{Q3}
\begin{frame}[t]{TV, Question 3}
% \vspace{0.5em}
\begin{block}{Question}
On which American game show did contestants answer questions in a taxi?
\end{block}
\end{frame}
\subsection*{Q4}
\begin{frame}[t]{TV, Question 4}
% \vspace{0.5em}
\begin{block}{Question}
What year did \emph{Saturday Night Live} first air?
\end{block}
\end{frame}
\subsection*{Q5}
\begin{frame}[t]{TV, Question 5}
% \vspace{0.5em}
\begin{block}{Question}
The actor who played Johnny Ola in \emph{The Godfather Part II} also played a character in \emph{The Sopranos}. Which \emph{Sopranos} character did he play?
\end{block}
\end{frame}
\subsection*{Q6}
\begin{frame}[t]{TV, Question 6}
% \vspace{0.5em}
\begin{block}{Question}
In \emph{The Office}, what is the title of the movie that it took Michael Scott ``three years of writing, one year of shooting, four years of re-shooting, and two years of editing'' to make?
\end{block}
\end{frame}
\subsection*{Q7}
\begin{frame}[t]{TV, Question 7}
% \vspace{0.5em}
\begin{block}{Question}
What was the first animated series made for primetime network TV\@?
\end{block}
\end{frame}
\subsection*{Q8}
\begin{frame}[t]{TV, Question 8}
% \vspace{0.5em}
\begin{block}{Question}
What was the first sitcom to write an actress's pregnancy into the storyline?
\end{block}
\end{frame}
\subsection*{Q9}
\begin{frame}[t]{TV, Question 9}
% \vspace{0.5em}
\begin{block}{Question}
Which HBO gangster show, which first aired in 2010, was set in Atlantic City?
\end{block}
\end{frame}
\subsection*{Q10}
\begin{frame}[t]{TV, Question 10}
% \vspace{0.5em}
\begin{block}{Question}
Which Spanish language comedy show, created by Don Francisco, ran in the U.S. from 1962 to 2015?
\end{block}
\end{frame}
\subsection{Answers}
\begin{frame}[t]{TV, Answer 1}
% \vspace{0.5em}
\begin{block}{Question}
On \emph{The Jeffersons}, George Jefferson owned a chain of what kind of businesses?
\end{block}

\visible<2->{
    \begin{columns}[T,totalwidth=\linewidth]
    \begin{column}{0.32\linewidth}
    \begin{block}{Answer}
    Dry cleaners
    \end{block}
    \end{column}
    \begin{column}{0.65\linewidth}
    \begin{center}
    \includegraphics[max width=0.95\textwidth,
        max height=0.54000\textheight]{{Images/georgejefferson}.jpg}
    \end{center}
    \end{column}
    \end{columns}
}
\end{frame}
\begin{frame}[t]{TV, Answer 2}
% \vspace{0.5em}
\begin{block}{Question}
Which TV comedy followed the production of the fictional TV show \emph{The Girlie Show}?
\end{block}

\visible<2->{
    \begin{columns}[T,totalwidth=\linewidth]
    \begin{column}{0.32\linewidth}
    \begin{block}{Answer}
    \emph{30 Rock}
    \end{block}
    \end{column}
    \begin{column}{0.65\linewidth}
    \begin{center}
    \includegraphics[max width=0.95\textwidth,
        max height=0.54000\textheight]{{Images/tgs}.png}
    \end{center}
    \end{column}
    \end{columns}
}
\end{frame}
\begin{frame}[t]{TV, Answer 3}
% \vspace{0.5em}
\begin{block}{Question}
On which American game show did contestants answer questions in a taxi?
\end{block}

\visible<2->{
    \begin{columns}[T,totalwidth=\linewidth]
    \begin{column}{0.32\linewidth}
    \begin{block}{Answer}
    \emph{Cash Cab} (on TV 2005 to 2012)
    \end{block}
    \end{column}
    \begin{column}{0.65\linewidth}
    \begin{center}
    \includegraphics[max width=0.95\textwidth,
        max height=0.54000\textheight]{{Images/cashcab}.jpg}
    \end{center}
    \end{column}
    \end{columns}
}
\end{frame}
\begin{frame}[t]{TV, Answer 4}
% \vspace{0.5em}
\begin{block}{Question}
What year did \emph{Saturday Night Live} first air?
\end{block}

\visible<2->{
    \begin{columns}[T,totalwidth=\linewidth]
    \begin{column}{0.32\linewidth}
    \begin{block}{Answer}
    1975
    \end{block}
    \end{column}
    \begin{column}{0.65\linewidth}
    \begin{center}
    \includegraphics[max width=0.95\textwidth,
        max height=0.54000\textheight]{{Images/snl}.jpg}
    \end{center}
    \end{column}
    \end{columns}
}
\end{frame}
\begin{frame}[t]{TV, Answer 5}
% \vspace{0.5em}
\begin{block}{Question}
The actor who played Johnny Ola in \emph{The Godfather Part II} also played a character in \emph{The Sopranos}. Which \emph{Sopranos} character did he play?
\end{block}

\visible<2->{
    \begin{columns}[T,totalwidth=\linewidth]
    \begin{column}{0.32\linewidth}
    \begin{block}{Answer}
    Corrado ``Junior'' Soprano, played by Dominic Chianese
    \end{block}
    \end{column}
    \begin{column}{0.65\linewidth}
    \begin{center}
    \includegraphics[max width=0.95\textwidth,
        max height=0.46000\textheight]{{Images/chianese}.jpg}
    \end{center}
    \end{column}
    \end{columns}
}
\end{frame}
\begin{frame}[t]{TV, Answer 6}
% \vspace{0.5em}
\begin{block}{Question}
In \emph{The Office}, what is the title of the movie that it took Michael Scott ``three years of writing, one year of shooting, four years of re-shooting, and two years of editing'' to make?
\end{block}

\visible<2->{
    \begin{columns}[T,totalwidth=\linewidth]
    \begin{column}{0.32\linewidth}
    \begin{block}{Answer}
    \emph{Threat Level Midnight}
    \end{block}
    \end{column}
    \begin{column}{0.65\linewidth}
    \begin{center}
    \includegraphics[max width=0.95\textwidth,
        max height=0.46000\textheight]{{Images/threatlevelmidnight}.jpg}
    \end{center}
    \end{column}
    \end{columns}
}
\end{frame}
\begin{frame}[t]{TV, Answer 7}
% \vspace{0.5em}
\begin{block}{Question}
What was the first animated series made for primetime network TV\@?
\end{block}

\visible<2->{
    \begin{columns}[T,totalwidth=\linewidth]
    \begin{column}{0.32\linewidth}
    \begin{block}{Answer}
    The Flintstones (on TV 1960 to 1966)
    \end{block}
    \end{column}
    \begin{column}{0.65\linewidth}
    \begin{center}
    \includegraphics[max width=0.95\textwidth,
        max height=0.54000\textheight]{{Images/flintstonescar}.jpg}
    \end{center}
    \end{column}
    \end{columns}
}
\end{frame}
\begin{frame}[t]{TV, Answer 8}
% \vspace{0.5em}
\begin{block}{Question}
What was the first sitcom to write an actress's pregnancy into the storyline?
\end{block}

\visible<2->{
    \begin{columns}[T,totalwidth=\linewidth]
    \begin{column}{0.32\linewidth}
    \begin{block}{Answer}
    \emph{Mary Kay and Johnny} (on TV 1947 to 1950)
    \end{block}
    \end{column}
    \begin{column}{0.65\linewidth}
    \begin{center}
    \includegraphics[max width=0.95\textwidth,
        max height=0.54000\textheight]{{Images/mkj}.jpg}
    \end{center}
    \end{column}
    \end{columns}
}
\end{frame}
\begin{frame}[t]{TV, Answer 9}
% \vspace{0.5em}
\begin{block}{Question}
Which HBO gangster show, which first aired in 2010, was set in Atlantic City?
\end{block}

\visible<2->{
    \begin{columns}[T,totalwidth=\linewidth]
    \begin{column}{0.32\linewidth}
    \begin{block}{Answer}
    \emph{Boardwalk Empire} (on TV 2010 to 2014)
    \end{block}
    \end{column}
    \begin{column}{0.65\linewidth}
    \begin{center}
    \includegraphics[max width=0.95\textwidth,
        max height=0.54000\textheight]{{Images/boardwalkempire}.jpg}
    \end{center}
    \end{column}
    \end{columns}
}
\end{frame}
\begin{frame}[t]{TV, Answer 10}
% \vspace{0.5em}
\begin{block}{Question}
Which Spanish language comedy show, created by Don Francisco, ran in the U.S. from 1962 to 2015?
\end{block}

\visible<2->{
    \begin{columns}[T,totalwidth=\linewidth]
    \begin{column}{0.32\linewidth}
    \begin{block}{Answer}
    \emph{Sabado Gigante}
    \end{block}
    \end{column}
    \begin{column}{0.65\linewidth}
    \begin{center}
    \includegraphics[max width=0.95\textwidth,
        max height=0.54000\textheight]{{Images/sabadogigante}.jpg}
    \end{center}
    \end{column}
    \end{columns}
}
\end{frame}
\def\thisSectionName{Bonus}
\section{Bonus Round}
\subsection*{Q1}
\begin{frame}[t]{Bonus Round: Logos}
% \vspace{0.5em}
\begin{columns}[T,totalwidth=\linewidth]
\begin{column}{0.32\linewidth}
\begin{block}{Question}
The barrels shown here were cropped from the logo of which soft drink company?
\end{block}
\end{column}
\begin{column}{0.65\linewidth}
\begin{center}
\includegraphics[max width=0.95\textwidth,max height=0.7\textheight]{{Images/barqsicon}.png}
\end{center}
\end{column}
\end{columns}
\end{frame}
\subsection*{Q2}
\begin{frame}[t]{Bonus Round: Cocktails}
% \vspace{0.5em}
\begin{block}{Question}
\begin{itemize}
\item 2\({}^1{\mskip -5mu⁄\mskip -3mu}_2\) oz.\ cognac
\item 1 oz.\ triple sec
\item 1 oz.\ lemon juice
\item Serve straight up
\end{itemize}
\end{block}
\end{frame}
\subsection*{Q3}
\begin{frame}[t]{Bonus Round: Superheroes}
% \vspace{0.5em}
\begin{block}{Question}
Captain America's first enemy belonged to which evil organization?
\end{block}
\end{frame}
\subsection*{Q4}
\begin{frame}[t]{Bonus Round: New York City}
% \vspace{0.5em}
\begin{block}{Question}
Name five of the seven New York City buildings completed after 1900 that were the tallest buildings in the world when they were completed. 
\end{block}
\end{frame}
\subsection*{Q5}
\begin{frame}[t]{Bonus Round: More Plants and Animals}
% \vspace{0.5em}
\begin{block}{Question}
Which mammal has the densest fur (measured in hairs per square inch)?
\end{block}
\end{frame}
\subsection*{Q6}
\begin{frame}[t]{Bonus Round: Geography}
% \vspace{0.5em}
\begin{block}{Question}
By volume, what is the largest freshwater lake in the world?
\end{block}
\end{frame}
\subsection*{Q7}
\begin{frame}[t]{Bonus Round: Real name/Stage name}
% \vspace{0.5em}
\begin{block}{Question}
What is the stage name of the actor born Lee Jun-fan (李振藩)?
\end{block}
\end{frame}
\subsection*{Q8}
\begin{frame}[t]{Bonus Round: What are they saying about me?}
% \vspace{0.5em}
\begin{block}{Question}
This famous split-up of a famous couple resulted in the filing of court papers in which one person, C\textunderscore{}\textunderscore{}\textunderscore{}\textunderscore{}\textunderscore{}, said of the other, B\textunderscore{}\textunderscore{}\textunderscore{}\textunderscore{}\textunderscore{}, that B\textunderscore{}\textunderscore{}\textunderscore{}\textunderscore{}\textunderscore{}, ``[N]ow accepts the reality of this situation and has advised this Court that he will consent to a dissolution\ldots{}.'' Who are C\textunderscore{}\textunderscore{}\textunderscore{}\textunderscore{}\textunderscore{} and B\textunderscore{}\textunderscore{}\textunderscore{}\textunderscore{}\textunderscore{}?
\end{block}
\end{frame}
\subsection*{Q9}
\begin{frame}[t]{Bonus Round: Ancient Civilizations}
% \vspace{0.5em}
\begin{block}{Question}
Which modern-day country has the most ancient pyramids of any country?
\end{block}
\end{frame}
\subsection*{Q10}
\begin{frame}[t]{Bonus Round: TV}
% \vspace{0.5em}
\begin{block}{Question}
Besides the ``Highlights of a Hundred'' ``best of'' show, there is only one episode of \emph{Seinfeld} whose title does not begin with the word ``The''. What is this episode's title?
\end{block}
\end{frame}
\subsection{Answers}
\begin{frame}[t]{Bonus Round: Logos}
% \vspace{0.5em}
\begin{columns}[T,totalwidth=\linewidth]
\begin{column}{0.32\linewidth}
\begin{block}{Question}
The barrels shown here were cropped from the logo of which soft drink company?
\end{block}
\end{column}
\begin{column}{0.65\linewidth}
\begin{center}
\includegraphics[max width=0.95\textwidth,max height=0.35\textheight]{{Images/barqsicon}.png}
\end{center}
\end{column}
\end{columns}

\visible<2->{
    \begin{columns}[T,totalwidth=\linewidth]
    \begin{column}{0.32\linewidth}
    \begin{block}{Answer}
    Barq's Root Beer
    \end{block}
    \end{column}
    \begin{column}{0.65\linewidth}
    \begin{center}
    \includegraphics[max width=0.95\textwidth,
        max height=0.38\textheight]{{Images/barqslogo}.png}
    \end{center}
    \end{column}
    \end{columns}
}
\end{frame}
\begin{frame}[t]{Bonus Round: Cocktails}
% \vspace{0.5em}
\begin{block}{Question}
\begin{itemize}
\item 2\({}^1{\mskip -5mu⁄\mskip -3mu}_2\) oz.\ cognac
\item 1 oz.\ triple sec
\item 1 oz.\ lemon juice
\item Serve straight up
\end{itemize}
\end{block}

\visible<2->{
    \begin{columns}[T,totalwidth=\linewidth]
    \begin{column}{0.32\linewidth}
    \begin{block}{Answer}
    Sidecar
    \end{block}
    \end{column}
    \begin{column}{0.65\linewidth}
    \begin{center}
    \includegraphics[max width=0.95\textwidth,
        max height=0.34000\textheight]{{Images/Sidecar}.jpg}
    \end{center}
    \end{column}
    \end{columns}
}
\end{frame}
\begin{frame}[t]{Bonus Round: Superheroes}
% \vspace{0.5em}
\begin{block}{Question}
Captain America's first enemy belonged to which evil organization?
\end{block}

\visible<2->{
    \begin{columns}[T,totalwidth=\linewidth]
    \begin{column}{0.32\linewidth}
    \begin{block}{Answer}
    The Nazis
    \end{block}
    \end{column}
    \begin{column}{0.65\linewidth}
    \begin{center}
    \includegraphics[max width=0.95\textwidth,
        max height=0.54000\textheight]{{Images/captainamerica}.jpg}
    \end{center}
    \end{column}
    \end{columns}
}
\end{frame}
\begin{frame}[t]{Bonus Round: New York City}
% \vspace{0.5em}
\begin{block}{Question}
Name five of the seven New York City buildings completed after 1900 that were the tallest buildings in the world when they were completed. 
\end{block}

\visible<2->{
    \begin{columns}[T,totalwidth=\linewidth]
    \begin{column}{0.5\linewidth}
    \begin{block}{Answer}
    Any five of these: Singer Building (1908), Met Life Tower (1909), Woolworth Building (1913), 40 Wall St.\ (1930), Chrysler Building (1930), Empire State Building (1931), World Trade Center (1973)
    \end{block}
    \end{column}
    \begin{column}{0.45\linewidth}
    \begin{center}
    \includegraphics[max width=0.95\textwidth,
        max height=0.7\textheight]{{Images/nycskyline}.jpg}
    \end{center}
    \end{column}
    \end{columns}
}
\end{frame}
\begin{frame}[t]{Bonus Round: More Plants and Animals}
% \vspace{0.5em}
\begin{block}{Question}
Which mammal has the densest fur (measured in hairs per square inch)?
\end{block}

\visible<2->{
    \begin{columns}[T,totalwidth=\linewidth]
    \begin{column}{0.32\linewidth}
    \begin{block}{Answer}
    Sea otters (\emph{Enhydra lutris}), at up to one million hairs per square inch
    \end{block}
    \end{column}
    \begin{column}{0.65\linewidth}
    \begin{center}
    \includegraphics[max width=0.95\textwidth,
        max height=0.54000\textheight]{{Images/otter}.jpg}
    \end{center}
    \end{column}
    \end{columns}
}
\end{frame}
\begin{frame}[t]{Bonus Round: Geography}
% \vspace{0.5em}
\begin{block}{Question}
By volume, what is the largest freshwater lake in the world?
\end{block}

\visible<2->{
    \begin{columns}[T,totalwidth=\linewidth]
    \begin{column}{0.32\linewidth}
    \begin{block}{Answer}
    Lake Baikal (located in Siberia and containing roughly 22\% of the world's surface freshwater) 
    \end{block}
    \end{column}
    \begin{column}{0.65\linewidth}
    \begin{center}
    \includegraphics[max width=0.95\textwidth,
        max height=0.54000\textheight]{{Images/baikal}.jpg}
    \end{center}
    \end{column}
    \end{columns}
}
\end{frame}
\begin{frame}[t]{Bonus Round: Real name/Stage name}
% \vspace{0.5em}
\begin{block}{Question}
What is the stage name of the actor born Lee Jun-fan (李振藩)?
\end{block}

\visible<2->{
    \begin{columns}[T,totalwidth=\linewidth]
    \begin{column}{0.32\linewidth}
    \begin{block}{Answer}
    Bruce Lee
    \end{block}
    \end{column}
    \begin{column}{0.65\linewidth}
    \begin{center}
    \includegraphics[max width=0.95\textwidth,
        max height=0.54000\textheight]{{Images/brucelee}.jpg}
    \end{center}
    \end{column}
    \end{columns}
}
\end{frame}
\begin{frame}[t]{Bonus Round: What are they saying about me?}
% \vspace{0.5em}
\begin{block}{Question}
This famous split-up of a famous couple resulted in the filing of court papers in which one person, C\textunderscore{}\textunderscore{}\textunderscore{}\textunderscore{}\textunderscore{}, said of the other, B\textunderscore{}\textunderscore{}\textunderscore{}\textunderscore{}\textunderscore{}, that B\textunderscore{}\textunderscore{}\textunderscore{}\textunderscore{}\textunderscore{}, ``[N]ow accepts the reality of this situation and has advised this Court that he will consent to a dissolution\ldots{}.'' Who are C\textunderscore{}\textunderscore{}\textunderscore{}\textunderscore{}\textunderscore{} and B\textunderscore{}\textunderscore{}\textunderscore{}\textunderscore{}\textunderscore{}?
\end{block}
\vspace{1em}
\pause{}
\begin{columns}[T,totalwidth=\linewidth]
\begin{column}{0.47\linewidth}
\begin{block}{Answer}
Cellino \& Barnes, Injury Attorneys, 800--888--8888
\end{block}
\end{column}
\begin{column}{0.47\linewidth}
\includegraphics[max width=0.95\textwidth,
        max height=0.4\textheight]{{Images/cellinobarnes}.jpg}
\end{column}
\end{columns}
\end{frame}
\begin{frame}[t]{Bonus Round: Ancient Civilizations}
% \vspace{0.5em}
\begin{block}{Question}
Which modern-day country has the most ancient pyramids of any country?
\end{block}

\visible<2->{
    \begin{columns}[T,totalwidth=\linewidth]
    \begin{column}{0.32\linewidth}
    \begin{block}{Answer}
    Sudan (the Nubian pyramids)
    \end{block}
    \end{column}
    \begin{column}{0.65\linewidth}
    \begin{center}
    \includegraphics[max width=0.95\textwidth,
        max height=0.54000\textheight]{{Images/pyramids}.jpg}
    \end{center}
    \end{column}
    \end{columns}
}
\end{frame}
\begin{frame}[t]{Bonus Round: TV}
% \vspace{0.5em}
\begin{block}{Question}
Besides the ``Highlights of a Hundred'' ``best of'' show, there is only one episode of \emph{Seinfeld} whose title does not begin with the word ``The''. What is this episode's title?
\end{block}

\visible<2->{
    \begin{columns}[T,totalwidth=\linewidth]
    \begin{column}{0.32\linewidth}
    \begin{block}{Answer}
    ``Male Unbonding'' (Season 1, Episode 4)
    \end{block}
    \end{column}
    \begin{column}{0.65\linewidth}
    \begin{center}
    \includegraphics[max width=0.95\textwidth,
        max height=0.46000\textheight]{{Images/maleunbonding}.jpg}
    \end{center}
    \end{column}
    \end{columns}
}
\end{frame}
\section*{\ }
\subsection*{\ }
\begingroup{}
\setbeamertemplate{headline}{}
\begin{frame}
\vfill{}
\centering{}
\begin{beamercolorbox}[sep=8pt,center,shadow=true,rounded=true]{title}
\usebeamerfont{title}Thanks for playing!\par%
\end{beamercolorbox}
\vfill{}
\end{frame}
\endgroup{}
% \begingroup{}
% \setbeamertemplate{headline}{}
% \section*{Thanks for playing!}
% \subsection*{\ }
% \endgroup{}

\end{document}